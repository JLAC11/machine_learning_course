\documentclass[a4paper,12pt]{amsart}

\usepackage[spanish]{babel}
\usepackage[style=ieee]{biblatex}
\addbibresource{./references.bib}
\usepackage[utf8]{inputenc}
\usepackage[margin=2cm]{geometry}
\usepackage[labelfont=bf,textfont=it]{caption,subcaption}
\usepackage{amsmath,amssymb,amsfonts}
\usepackage{graphicx,fancyhdr}
\usepackage{csquotes}

\author{José Luis Aguilar Charfén}
\date{\today}
\title{Objetivos de Desarrollo Sostenible: proyecto de Machine Learning}




\begin{document}
    \begin{abstract} % ! TODO
        Este reporte muestra tanto un agrupamiento de los países de acuerdo a 
        las métricas utilizadas para conocer el progreso en las 17 metas de 
        sustentabilidad que proponen las Naciones unidas, como en estos mismos 
        indicadores. A su vez, también se propone una manera de evaluar el progreso 
        de los países en temas de sustentabilidad. 
    \end{abstract}
    \maketitle

    \section{Introducción}

    Los Objetivos de Desarrollo Sustentable son 17 metas que se acordaron en la 
    Organización de las Naciones Unidas (ONU) en 2015 \cite{united_nations_development_programme_sustainable_nodate-1}
    como una llamada a la acción para proteger al planeta y asegurar que para
    2030 la población disfrute de paz y seguridad. 

    Estos objetivos varían desde reducir la pobreza, hasta temas de salud, 
    equidad de género, higiene, crecimiento económico, energía y cuidado al medio ambiente. 
    Cuando se consideran todos los diferentes objetivos, entonces se construye el 
    SDG Index Score, que califica el progreso de un país en las 17 metas. Para 
    este reporte, se tomarán tanto las calificaciones otorgadas por la ONU, como
    los datos a partir de los cuales se construyen las calificaciones en cada 
    uno de los criterios.

    Se mostrará en primera instancia un agrupamiento jerárquico para determinar 
    qué países son más cercanos en ambos rubros discutidos, y posteriormente, se 
    discutirá brevemente una manera de evaluar el progreso de los países en temas 
    de sustentabilidad. 
    		
    \section{Metodología}

    Los datos utilizados son obtenidos de \cite{united_nations_development_programme_sustainable_nodate}. 
    El algoritmo utilizado puede dividirse en las siguientes partes: 

    \begin{enumerate}
        \item Escalamiento.
        \item Imputación.
        \item Reducción de dimensionalidad.
        \item Agrupamiento.
    \end{enumerate}

    Debido a que no todos los países tienen un SDG Index Score, no se consideran
    para entrenamiento ni validación ni prueba a estos países. Sin embargo, sí 
    se predicen sus valores dadas las variables usadas para construir los primeros 
    dos SDGs.

    Para construir el modelo, se realiza un análisis de correlación de variables, 
    y se observan tanto las distribuciones como las correlaciones con gráficos 
    por pares de variables. Posteriormente, se transforman las variables con la 
    transformación Yeo-Johnson \cite{yeo_new_2000}, preferida sobre la de Box y Cox por la presencia 
    de valores 0 en las variables. Posteriormente, se imputan los valores con 
    metodología de vecinos más cercanos \cite{beretta_nearest_2016}. Después, 
    se codifican las regiones usadas para el SDG Index por un método one-hot. 
    Una vez transformadas las variables de estas maneras, se comparan un modelo 
    de mínimos cuadrados ordinarios, regresión Ridge, y Lasso, ajustando sus 
    regularizaciones por validación cruzada, y se evalúa si se encuentra 
    sobreajustado comparando el error cuadrático medio del conjunto de datos 
    de entrenamiento con uno de prueba por separado, corriendo 20 veces el 
    experimento y promediando los valores.

    Las decisiones de construcción del modelo y justificaciones se discuten a
    mayor profundidad en la sección de resultados y conclusión.

    \section{Resultados y discusión}
    

    \section{Conclusiones}

    \section{Referencias}
    \printbibliography

\end{document}