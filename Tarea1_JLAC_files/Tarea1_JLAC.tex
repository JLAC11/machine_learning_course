\documentclass[11pt]{article}

    \usepackage[breakable]{tcolorbox}
    \usepackage{parskip} % Stop auto-indenting (to mimic markdown behaviour)
    
    \usepackage{iftex}
    \ifPDFTeX
    	\usepackage[T1]{fontenc}
    	\usepackage{mathpazo}
    \else
    	\usepackage{fontspec}
    \fi

    % Basic figure setup, for now with no caption control since it's done
    % automatically by Pandoc (which extracts ![](path) syntax from Markdown).
    \usepackage{graphicx}
    % Maintain compatibility with old templates. Remove in nbconvert 6.0
    \let\Oldincludegraphics\includegraphics
    % Ensure that by default, figures have no caption (until we provide a
    % proper Figure object with a Caption API and a way to capture that
    % in the conversion process - todo).
    \usepackage{caption}
    \DeclareCaptionFormat{nocaption}{}
    \captionsetup{format=nocaption,aboveskip=0pt,belowskip=0pt}

    \usepackage{float}
    \floatplacement{figure}{H} % forces figures to be placed at the correct location
    \usepackage{xcolor} % Allow colors to be defined
    \usepackage{enumerate} % Needed for markdown enumerations to work
    \usepackage{geometry} % Used to adjust the document margins
    \usepackage{amsmath} % Equations
    \usepackage{amssymb} % Equations
    \usepackage{textcomp} % defines textquotesingle
    % Hack from http://tex.stackexchange.com/a/47451/13684:
    \AtBeginDocument{%
        \def\PYZsq{\textquotesingle}% Upright quotes in Pygmentized code
    }
    \usepackage[spanish]{babel}
    \usepackage{upquote} % Upright quotes for verbatim code
    \usepackage{eurosym} % defines \euro
    \usepackage[mathletters]{ucs} % Extended unicode (utf-8) support
    \usepackage{fancyvrb} % verbatim replacement that allows latex
    \usepackage{grffile} % extends the file name processing of package graphics 
                         % to support a larger range
    \makeatletter % fix for old versions of grffile with XeLaTeX
    \@ifpackagelater{grffile}{2019/11/01}
    {
      % Do nothing on new versions
    }
    {
      \def\Gread@@xetex#1{%
        \IfFileExists{"\Gin@base".bb}%
        {\Gread@eps{\Gin@base.bb}}%
        {\Gread@@xetex@aux#1}%
      }
    }
    \makeatother
    \usepackage[Export]{adjustbox} % Used to constrain images to a maximum size
    \adjustboxset{max size={0.9\linewidth}{0.9\paperheight}}

    % The hyperref package gives us a pdf with properly built
    % internal navigation ('pdf bookmarks' for the table of contents,
    % internal cross-reference links, web links for URLs, etc.)
    \usepackage{hyperref}
    % The default LaTeX title has an obnoxious amount of whitespace. By default,
    % titling removes some of it. It also provides customization options.
    \usepackage{titling}
    \usepackage{longtable} % longtable support required by pandoc >1.10
    \usepackage{booktabs}  % table support for pandoc > 1.12.2
    \usepackage[inline]{enumitem} % IRkernel/repr support (it uses the enumerate* environment)
    \usepackage[normalem]{ulem} % ulem is needed to support strikethroughs (\sout)
                                % normalem makes italics be italics, not underlines
    \usepackage{mathrsfs}
    

    
    % Colors for the hyperref package
    \definecolor{urlcolor}{rgb}{0,.145,.698}
    \definecolor{linkcolor}{rgb}{.71,0.21,0.01}
    \definecolor{citecolor}{rgb}{.12,.54,.11}

    % ANSI colors
    \definecolor{ansi-black}{HTML}{3E424D}
    \definecolor{ansi-black-intense}{HTML}{282C36}
    \definecolor{ansi-red}{HTML}{E75C58}
    \definecolor{ansi-red-intense}{HTML}{B22B31}
    \definecolor{ansi-green}{HTML}{00A250}
    \definecolor{ansi-green-intense}{HTML}{007427}
    \definecolor{ansi-yellow}{HTML}{DDB62B}
    \definecolor{ansi-yellow-intense}{HTML}{B27D12}
    \definecolor{ansi-blue}{HTML}{208FFB}
    \definecolor{ansi-blue-intense}{HTML}{0065CA}
    \definecolor{ansi-magenta}{HTML}{D160C4}
    \definecolor{ansi-magenta-intense}{HTML}{A03196}
    \definecolor{ansi-cyan}{HTML}{60C6C8}
    \definecolor{ansi-cyan-intense}{HTML}{258F8F}
    \definecolor{ansi-white}{HTML}{C5C1B4}
    \definecolor{ansi-white-intense}{HTML}{A1A6B2}
    \definecolor{ansi-default-inverse-fg}{HTML}{FFFFFF}
    \definecolor{ansi-default-inverse-bg}{HTML}{000000}

    % common color for the border for error outputs.
    \definecolor{outerrorbackground}{HTML}{FFDFDF}

    % commands and environments needed by pandoc snippets
    % extracted from the output of `pandoc -s`
    \providecommand{\tightlist}{%
      \setlength{\itemsep}{0pt}\setlength{\parskip}{0pt}}
    \DefineVerbatimEnvironment{Highlighting}{Verbatim}{commandchars=\\\{\}}
    % Add ',fontsize=\small' for more characters per line
    \newenvironment{Shaded}{}{}
    \newcommand{\KeywordTok}[1]{\textcolor[rgb]{0.00,0.44,0.13}{\textbf{{#1}}}}
    \newcommand{\DataTypeTok}[1]{\textcolor[rgb]{0.56,0.13,0.00}{{#1}}}
    \newcommand{\DecValTok}[1]{\textcolor[rgb]{0.25,0.63,0.44}{{#1}}}
    \newcommand{\BaseNTok}[1]{\textcolor[rgb]{0.25,0.63,0.44}{{#1}}}
    \newcommand{\FloatTok}[1]{\textcolor[rgb]{0.25,0.63,0.44}{{#1}}}
    \newcommand{\CharTok}[1]{\textcolor[rgb]{0.25,0.44,0.63}{{#1}}}
    \newcommand{\StringTok}[1]{\textcolor[rgb]{0.25,0.44,0.63}{{#1}}}
    \newcommand{\CommentTok}[1]{\textcolor[rgb]{0.38,0.63,0.69}{\textit{{#1}}}}
    \newcommand{\OtherTok}[1]{\textcolor[rgb]{0.00,0.44,0.13}{{#1}}}
    \newcommand{\AlertTok}[1]{\textcolor[rgb]{1.00,0.00,0.00}{\textbf{{#1}}}}
    \newcommand{\FunctionTok}[1]{\textcolor[rgb]{0.02,0.16,0.49}{{#1}}}
    \newcommand{\RegionMarkerTok}[1]{{#1}}
    \newcommand{\ErrorTok}[1]{\textcolor[rgb]{1.00,0.00,0.00}{\textbf{{#1}}}}
    \newcommand{\NormalTok}[1]{{#1}}
    
    % Additional commands for more recent versions of Pandoc
    \newcommand{\ConstantTok}[1]{\textcolor[rgb]{0.53,0.00,0.00}{{#1}}}
    \newcommand{\SpecialCharTok}[1]{\textcolor[rgb]{0.25,0.44,0.63}{{#1}}}
    \newcommand{\VerbatimStringTok}[1]{\textcolor[rgb]{0.25,0.44,0.63}{{#1}}}
    \newcommand{\SpecialStringTok}[1]{\textcolor[rgb]{0.73,0.40,0.53}{{#1}}}
    \newcommand{\ImportTok}[1]{{#1}}
    \newcommand{\DocumentationTok}[1]{\textcolor[rgb]{0.73,0.13,0.13}{\textit{{#1}}}}
    \newcommand{\AnnotationTok}[1]{\textcolor[rgb]{0.38,0.63,0.69}{\textbf{\textit{{#1}}}}}
    \newcommand{\CommentVarTok}[1]{\textcolor[rgb]{0.38,0.63,0.69}{\textbf{\textit{{#1}}}}}
    \newcommand{\VariableTok}[1]{\textcolor[rgb]{0.10,0.09,0.49}{{#1}}}
    \newcommand{\ControlFlowTok}[1]{\textcolor[rgb]{0.00,0.44,0.13}{\textbf{{#1}}}}
    \newcommand{\OperatorTok}[1]{\textcolor[rgb]{0.40,0.40,0.40}{{#1}}}
    \newcommand{\BuiltInTok}[1]{{#1}}
    \newcommand{\ExtensionTok}[1]{{#1}}
    \newcommand{\PreprocessorTok}[1]{\textcolor[rgb]{0.74,0.48,0.00}{{#1}}}
    \newcommand{\AttributeTok}[1]{\textcolor[rgb]{0.49,0.56,0.16}{{#1}}}
    \newcommand{\InformationTok}[1]{\textcolor[rgb]{0.38,0.63,0.69}{\textbf{\textit{{#1}}}}}
    \newcommand{\WarningTok}[1]{\textcolor[rgb]{0.38,0.63,0.69}{\textbf{\textit{{#1}}}}}
    
    
    % Define a nice break command that doesn't care if a line doesn't already
    % exist.
    \def\br{\hspace*{\fill} \\* }
    % Math Jax compatibility definitions
    \def\gt{>}
    \def\lt{<}
    \let\Oldtex\TeX
    \let\Oldlatex\LaTeX
    \renewcommand{\TeX}{\textrm{\Oldtex}}
    \renewcommand{\LaTeX}{\textrm{\Oldlatex}}
    % Document parameters
    % Document title
    \title{Tarea 1: Machine Learning}
    
    
    
    \author{José Luis Aguilar}
    
    
    
% Pygments definitions
\makeatletter
\def\PY@reset{\let\PY@it=\relax \let\PY@bf=\relax%
    \let\PY@ul=\relax \let\PY@tc=\relax%
    \let\PY@bc=\relax \let\PY@ff=\relax}
\def\PY@tok#1{\csname PY@tok@#1\endcsname}
\def\PY@toks#1+{\ifx\relax#1\empty\else%
    \PY@tok{#1}\expandafter\PY@toks\fi}
\def\PY@do#1{\PY@bc{\PY@tc{\PY@ul{%
    \PY@it{\PY@bf{\PY@ff{#1}}}}}}}
\def\PY#1#2{\PY@reset\PY@toks#1+\relax+\PY@do{#2}}

\@namedef{PY@tok@w}{\def\PY@tc##1{\textcolor[rgb]{0.73,0.73,0.73}{##1}}}
\@namedef{PY@tok@c}{\let\PY@it=\textit\def\PY@tc##1{\textcolor[rgb]{0.25,0.50,0.50}{##1}}}
\@namedef{PY@tok@cp}{\def\PY@tc##1{\textcolor[rgb]{0.74,0.48,0.00}{##1}}}
\@namedef{PY@tok@k}{\let\PY@bf=\textbf\def\PY@tc##1{\textcolor[rgb]{0.00,0.50,0.00}{##1}}}
\@namedef{PY@tok@kp}{\def\PY@tc##1{\textcolor[rgb]{0.00,0.50,0.00}{##1}}}
\@namedef{PY@tok@kt}{\def\PY@tc##1{\textcolor[rgb]{0.69,0.00,0.25}{##1}}}
\@namedef{PY@tok@o}{\def\PY@tc##1{\textcolor[rgb]{0.40,0.40,0.40}{##1}}}
\@namedef{PY@tok@ow}{\let\PY@bf=\textbf\def\PY@tc##1{\textcolor[rgb]{0.67,0.13,1.00}{##1}}}
\@namedef{PY@tok@nb}{\def\PY@tc##1{\textcolor[rgb]{0.00,0.50,0.00}{##1}}}
\@namedef{PY@tok@nf}{\def\PY@tc##1{\textcolor[rgb]{0.00,0.00,1.00}{##1}}}
\@namedef{PY@tok@nc}{\let\PY@bf=\textbf\def\PY@tc##1{\textcolor[rgb]{0.00,0.00,1.00}{##1}}}
\@namedef{PY@tok@nn}{\let\PY@bf=\textbf\def\PY@tc##1{\textcolor[rgb]{0.00,0.00,1.00}{##1}}}
\@namedef{PY@tok@ne}{\let\PY@bf=\textbf\def\PY@tc##1{\textcolor[rgb]{0.82,0.25,0.23}{##1}}}
\@namedef{PY@tok@nv}{\def\PY@tc##1{\textcolor[rgb]{0.10,0.09,0.49}{##1}}}
\@namedef{PY@tok@no}{\def\PY@tc##1{\textcolor[rgb]{0.53,0.00,0.00}{##1}}}
\@namedef{PY@tok@nl}{\def\PY@tc##1{\textcolor[rgb]{0.63,0.63,0.00}{##1}}}
\@namedef{PY@tok@ni}{\let\PY@bf=\textbf\def\PY@tc##1{\textcolor[rgb]{0.60,0.60,0.60}{##1}}}
\@namedef{PY@tok@na}{\def\PY@tc##1{\textcolor[rgb]{0.49,0.56,0.16}{##1}}}
\@namedef{PY@tok@nt}{\let\PY@bf=\textbf\def\PY@tc##1{\textcolor[rgb]{0.00,0.50,0.00}{##1}}}
\@namedef{PY@tok@nd}{\def\PY@tc##1{\textcolor[rgb]{0.67,0.13,1.00}{##1}}}
\@namedef{PY@tok@s}{\def\PY@tc##1{\textcolor[rgb]{0.73,0.13,0.13}{##1}}}
\@namedef{PY@tok@sd}{\let\PY@it=\textit\def\PY@tc##1{\textcolor[rgb]{0.73,0.13,0.13}{##1}}}
\@namedef{PY@tok@si}{\let\PY@bf=\textbf\def\PY@tc##1{\textcolor[rgb]{0.73,0.40,0.53}{##1}}}
\@namedef{PY@tok@se}{\let\PY@bf=\textbf\def\PY@tc##1{\textcolor[rgb]{0.73,0.40,0.13}{##1}}}
\@namedef{PY@tok@sr}{\def\PY@tc##1{\textcolor[rgb]{0.73,0.40,0.53}{##1}}}
\@namedef{PY@tok@ss}{\def\PY@tc##1{\textcolor[rgb]{0.10,0.09,0.49}{##1}}}
\@namedef{PY@tok@sx}{\def\PY@tc##1{\textcolor[rgb]{0.00,0.50,0.00}{##1}}}
\@namedef{PY@tok@m}{\def\PY@tc##1{\textcolor[rgb]{0.40,0.40,0.40}{##1}}}
\@namedef{PY@tok@gh}{\let\PY@bf=\textbf\def\PY@tc##1{\textcolor[rgb]{0.00,0.00,0.50}{##1}}}
\@namedef{PY@tok@gu}{\let\PY@bf=\textbf\def\PY@tc##1{\textcolor[rgb]{0.50,0.00,0.50}{##1}}}
\@namedef{PY@tok@gd}{\def\PY@tc##1{\textcolor[rgb]{0.63,0.00,0.00}{##1}}}
\@namedef{PY@tok@gi}{\def\PY@tc##1{\textcolor[rgb]{0.00,0.63,0.00}{##1}}}
\@namedef{PY@tok@gr}{\def\PY@tc##1{\textcolor[rgb]{1.00,0.00,0.00}{##1}}}
\@namedef{PY@tok@ge}{\let\PY@it=\textit}
\@namedef{PY@tok@gs}{\let\PY@bf=\textbf}
\@namedef{PY@tok@gp}{\let\PY@bf=\textbf\def\PY@tc##1{\textcolor[rgb]{0.00,0.00,0.50}{##1}}}
\@namedef{PY@tok@go}{\def\PY@tc##1{\textcolor[rgb]{0.53,0.53,0.53}{##1}}}
\@namedef{PY@tok@gt}{\def\PY@tc##1{\textcolor[rgb]{0.00,0.27,0.87}{##1}}}
\@namedef{PY@tok@err}{\def\PY@bc##1{{\setlength{\fboxsep}{\string -\fboxrule}\fcolorbox[rgb]{1.00,0.00,0.00}{1,1,1}{\strut ##1}}}}
\@namedef{PY@tok@kc}{\let\PY@bf=\textbf\def\PY@tc##1{\textcolor[rgb]{0.00,0.50,0.00}{##1}}}
\@namedef{PY@tok@kd}{\let\PY@bf=\textbf\def\PY@tc##1{\textcolor[rgb]{0.00,0.50,0.00}{##1}}}
\@namedef{PY@tok@kn}{\let\PY@bf=\textbf\def\PY@tc##1{\textcolor[rgb]{0.00,0.50,0.00}{##1}}}
\@namedef{PY@tok@kr}{\let\PY@bf=\textbf\def\PY@tc##1{\textcolor[rgb]{0.00,0.50,0.00}{##1}}}
\@namedef{PY@tok@bp}{\def\PY@tc##1{\textcolor[rgb]{0.00,0.50,0.00}{##1}}}
\@namedef{PY@tok@fm}{\def\PY@tc##1{\textcolor[rgb]{0.00,0.00,1.00}{##1}}}
\@namedef{PY@tok@vc}{\def\PY@tc##1{\textcolor[rgb]{0.10,0.09,0.49}{##1}}}
\@namedef{PY@tok@vg}{\def\PY@tc##1{\textcolor[rgb]{0.10,0.09,0.49}{##1}}}
\@namedef{PY@tok@vi}{\def\PY@tc##1{\textcolor[rgb]{0.10,0.09,0.49}{##1}}}
\@namedef{PY@tok@vm}{\def\PY@tc##1{\textcolor[rgb]{0.10,0.09,0.49}{##1}}}
\@namedef{PY@tok@sa}{\def\PY@tc##1{\textcolor[rgb]{0.73,0.13,0.13}{##1}}}
\@namedef{PY@tok@sb}{\def\PY@tc##1{\textcolor[rgb]{0.73,0.13,0.13}{##1}}}
\@namedef{PY@tok@sc}{\def\PY@tc##1{\textcolor[rgb]{0.73,0.13,0.13}{##1}}}
\@namedef{PY@tok@dl}{\def\PY@tc##1{\textcolor[rgb]{0.73,0.13,0.13}{##1}}}
\@namedef{PY@tok@s2}{\def\PY@tc##1{\textcolor[rgb]{0.73,0.13,0.13}{##1}}}
\@namedef{PY@tok@sh}{\def\PY@tc##1{\textcolor[rgb]{0.73,0.13,0.13}{##1}}}
\@namedef{PY@tok@s1}{\def\PY@tc##1{\textcolor[rgb]{0.73,0.13,0.13}{##1}}}
\@namedef{PY@tok@mb}{\def\PY@tc##1{\textcolor[rgb]{0.40,0.40,0.40}{##1}}}
\@namedef{PY@tok@mf}{\def\PY@tc##1{\textcolor[rgb]{0.40,0.40,0.40}{##1}}}
\@namedef{PY@tok@mh}{\def\PY@tc##1{\textcolor[rgb]{0.40,0.40,0.40}{##1}}}
\@namedef{PY@tok@mi}{\def\PY@tc##1{\textcolor[rgb]{0.40,0.40,0.40}{##1}}}
\@namedef{PY@tok@il}{\def\PY@tc##1{\textcolor[rgb]{0.40,0.40,0.40}{##1}}}
\@namedef{PY@tok@mo}{\def\PY@tc##1{\textcolor[rgb]{0.40,0.40,0.40}{##1}}}
\@namedef{PY@tok@ch}{\let\PY@it=\textit\def\PY@tc##1{\textcolor[rgb]{0.25,0.50,0.50}{##1}}}
\@namedef{PY@tok@cm}{\let\PY@it=\textit\def\PY@tc##1{\textcolor[rgb]{0.25,0.50,0.50}{##1}}}
\@namedef{PY@tok@cpf}{\let\PY@it=\textit\def\PY@tc##1{\textcolor[rgb]{0.25,0.50,0.50}{##1}}}
\@namedef{PY@tok@c1}{\let\PY@it=\textit\def\PY@tc##1{\textcolor[rgb]{0.25,0.50,0.50}{##1}}}
\@namedef{PY@tok@cs}{\let\PY@it=\textit\def\PY@tc##1{\textcolor[rgb]{0.25,0.50,0.50}{##1}}}

\def\PYZbs{\char`\\}
\def\PYZus{\char`\_}
\def\PYZob{\char`\{}
\def\PYZcb{\char`\}}
\def\PYZca{\char`\^}
\def\PYZam{\char`\&}
\def\PYZlt{\char`\<}
\def\PYZgt{\char`\>}
\def\PYZsh{\char`\#}
\def\PYZpc{\char`\%}
\def\PYZdl{\char`\$}
\def\PYZhy{\char`\-}
\def\PYZsq{\char`\'}
\def\PYZdq{\char`\"}
\def\PYZti{\char`\~}
% for compatibility with earlier versions
\def\PYZat{@}
\def\PYZlb{[}
\def\PYZrb{]}
\makeatother


    % For linebreaks inside Verbatim environment from package fancyvrb. 
    \makeatletter
        \newbox\Wrappedcontinuationbox 
        \newbox\Wrappedvisiblespacebox 
        \newcommand*\Wrappedvisiblespace {\textcolor{red}{\textvisiblespace}} 
        \newcommand*\Wrappedcontinuationsymbol {\textcolor{red}{\llap{\tiny$\m@th\hookrightarrow$}}} 
        \newcommand*\Wrappedcontinuationindent {3ex } 
        \newcommand*\Wrappedafterbreak {\kern\Wrappedcontinuationindent\copy\Wrappedcontinuationbox} 
        % Take advantage of the already applied Pygments mark-up to insert 
        % potential linebreaks for TeX processing. 
        %        {, <, #, %, $, ' and ": go to next line. 
        %        _, }, ^, &, >, - and ~: stay at end of broken line. 
        % Use of \textquotesingle for straight quote. 
        \newcommand*\Wrappedbreaksatspecials {% 
            \def\PYGZus{\discretionary{\char`\_}{\Wrappedafterbreak}{\char`\_}}% 
            \def\PYGZob{\discretionary{}{\Wrappedafterbreak\char`\{}{\char`\{}}% 
            \def\PYGZcb{\discretionary{\char`\}}{\Wrappedafterbreak}{\char`\}}}% 
            \def\PYGZca{\discretionary{\char`\^}{\Wrappedafterbreak}{\char`\^}}% 
            \def\PYGZam{\discretionary{\char`\&}{\Wrappedafterbreak}{\char`\&}}% 
            \def\PYGZlt{\discretionary{}{\Wrappedafterbreak\char`\<}{\char`\<}}% 
            \def\PYGZgt{\discretionary{\char`\>}{\Wrappedafterbreak}{\char`\>}}% 
            \def\PYGZsh{\discretionary{}{\Wrappedafterbreak\char`\#}{\char`\#}}% 
            \def\PYGZpc{\discretionary{}{\Wrappedafterbreak\char`\%}{\char`\%}}% 
            \def\PYGZdl{\discretionary{}{\Wrappedafterbreak\char`\$}{\char`\$}}% 
            \def\PYGZhy{\discretionary{\char`\-}{\Wrappedafterbreak}{\char`\-}}% 
            \def\PYGZsq{\discretionary{}{\Wrappedafterbreak\textquotesingle}{\textquotesingle}}% 
            \def\PYGZdq{\discretionary{}{\Wrappedafterbreak\char`\"}{\char`\"}}% 
            \def\PYGZti{\discretionary{\char`\~}{\Wrappedafterbreak}{\char`\~}}% 
        } 
        % Some characters . , ; ? ! / are not pygmentized. 
        % This macro makes them "active" and they will insert potential linebreaks 
        \newcommand*\Wrappedbreaksatpunct {% 
            \lccode`\~`\.\lowercase{\def~}{\discretionary{\hbox{\char`\.}}{\Wrappedafterbreak}{\hbox{\char`\.}}}% 
            \lccode`\~`\,\lowercase{\def~}{\discretionary{\hbox{\char`\,}}{\Wrappedafterbreak}{\hbox{\char`\,}}}% 
            \lccode`\~`\;\lowercase{\def~}{\discretionary{\hbox{\char`\;}}{\Wrappedafterbreak}{\hbox{\char`\;}}}% 
            \lccode`\~`\:\lowercase{\def~}{\discretionary{\hbox{\char`\:}}{\Wrappedafterbreak}{\hbox{\char`\:}}}% 
            \lccode`\~`\?\lowercase{\def~}{\discretionary{\hbox{\char`\?}}{\Wrappedafterbreak}{\hbox{\char`\?}}}% 
            \lccode`\~`\!\lowercase{\def~}{\discretionary{\hbox{\char`\!}}{\Wrappedafterbreak}{\hbox{\char`\!}}}% 
            \lccode`\~`\/\lowercase{\def~}{\discretionary{\hbox{\char`\/}}{\Wrappedafterbreak}{\hbox{\char`\/}}}% 
            \catcode`\.\active
            \catcode`\,\active 
            \catcode`\;\active
            \catcode`\:\active
            \catcode`\?\active
            \catcode`\!\active
            \catcode`\/\active 
            \lccode`\~`\~ 	
        }
    \makeatother

    \let\OriginalVerbatim=\Verbatim
    \makeatletter
    \renewcommand{\Verbatim}[1][1]{%
        %\parskip\z@skip
        \sbox\Wrappedcontinuationbox {\Wrappedcontinuationsymbol}%
        \sbox\Wrappedvisiblespacebox {\FV@SetupFont\Wrappedvisiblespace}%
        \def\FancyVerbFormatLine ##1{\hsize\linewidth
            \vtop{\raggedright\hyphenpenalty\z@\exhyphenpenalty\z@
                \doublehyphendemerits\z@\finalhyphendemerits\z@
                \strut ##1\strut}%
        }%
        % If the linebreak is at a space, the latter will be displayed as visible
        % space at end of first line, and a continuation symbol starts next line.
        % Stretch/shrink are however usually zero for typewriter font.
        \def\FV@Space {%
            \nobreak\hskip\z@ plus\fontdimen3\font minus\fontdimen4\font
            \discretionary{\copy\Wrappedvisiblespacebox}{\Wrappedafterbreak}
            {\kern\fontdimen2\font}%
        }%
        
        % Allow breaks at special characters using \PYG... macros.
        \Wrappedbreaksatspecials
        % Breaks at punctuation characters . , ; ? ! and / need catcode=\active 	
        \OriginalVerbatim[#1,codes*=\Wrappedbreaksatpunct]%
    }
    \makeatother

    % Exact colors from NB
    \definecolor{incolor}{HTML}{303F9F}
    \definecolor{outcolor}{HTML}{D84315}
    \definecolor{cellborder}{HTML}{CFCFCF}
    \definecolor{cellbackground}{HTML}{F7F7F7}
    
    % prompt
    \makeatletter
    \newcommand{\boxspacing}{\kern\kvtcb@left@rule\kern\kvtcb@boxsep}
    \makeatother
    \newcommand{\prompt}[4]{
        {\ttfamily\llap{{\color{#2}[#3]:\hspace{3pt}#4}}\vspace{-\baselineskip}}
    }
    

    
    % Prevent overflowing lines due to hard-to-break entities
    \sloppy 
    % Setup hyperref package
    \hypersetup{
      breaklinks=true,  % so long urls are correctly broken across lines
      colorlinks=true,
      urlcolor=urlcolor,
      linkcolor=linkcolor,
      citecolor=citecolor,
      }
    % Slightly bigger margins than the latex defaults
    
    \geometry{verbose,tmargin=1in,bmargin=1in,lmargin=1in,rmargin=1in}
    
    

\begin{document}
    
    \maketitle
    
    

    
    Para la siguiente tarea, se utilizan las siguientes bibliotecas en
Python, que se importan de una vez:

    \begin{tcolorbox}[breakable, size=fbox, boxrule=1pt, pad at break*=1mm,colback=cellbackground, colframe=cellborder]
\prompt{In}{incolor}{1}{\boxspacing}
\begin{Verbatim}[commandchars=\\\{\}]
\PY{k+kn}{import} \PY{n+nn}{pandas} \PY{k}{as} \PY{n+nn}{pd}
\PY{k+kn}{import} \PY{n+nn}{numpy} \PY{k}{as} \PY{n+nn}{np} 
\PY{k+kn}{import} \PY{n+nn}{statsmodels}\PY{n+nn}{.}\PY{n+nn}{api} \PY{k}{as} \PY{n+nn}{sm}
\PY{k+kn}{import} \PY{n+nn}{seaborn} \PY{k}{as} \PY{n+nn}{sns}
\PY{k+kn}{import} \PY{n+nn}{matplotlib}\PY{n+nn}{.}\PY{n+nn}{pyplot} \PY{k}{as} \PY{n+nn}{plt}
\PY{k+kn}{from} \PY{n+nn}{sklearn}\PY{n+nn}{.}\PY{n+nn}{preprocessing} \PY{k+kn}{import} \PY{n}{StandardScaler}\PY{p}{,} \PY{n}{PowerTransformer}
\PY{k+kn}{from} \PY{n+nn}{sklearn}\PY{n+nn}{.}\PY{n+nn}{linear\PYZus{}model} \PY{k+kn}{import} \PY{n}{Ridge}\PY{p}{,} \PY{n}{RidgeCV}
\PY{k+kn}{from} \PY{n+nn}{sklearn}\PY{n+nn}{.}\PY{n+nn}{model\PYZus{}selection} \PY{k+kn}{import} \PY{n}{KFold}
\PY{k+kn}{from} \PY{n+nn}{sklearn}\PY{n+nn}{.}\PY{n+nn}{metrics} \PY{k+kn}{import} \PY{n}{mean\PYZus{}squared\PYZus{}error}
\PY{n}{sns}\PY{o}{.}\PY{n}{set\PYZus{}theme}\PY{p}{(}\PY{p}{)}
\end{Verbatim}
\end{tcolorbox}

    \hypertarget{ejercicio-1}{%
\subsection{Ejercicio 1}\label{ejercicio-1}}

Muestre que el estimador de regresión ridge se puede obtener por mínimos
cuadrados usuales agregado \(p\) filas adicionales
\(\sqrt{\lambda} \mathbf{I}_{pp}\) a la matriz \(\mathbf{X}\) y
agregando \(p\) ceros al vector de respuestas \(\mathbf{Y}\).

    \hypertarget{respuesta}{%
\subsubsection{Respuesta}\label{respuesta}}

Sea \(\mathbf{X}\) una matriz de tamaño \(n\times p\) de predictores,
\(\mathbf{Y}\) un vector columna de tamaño \(n\) de variables de
respuesta, \(\mathbf{X}_r\) la matriz \(\mathbf{X}\) con las filas
adicionales con forma \(\sqrt{\lambda} \mathbf{I}_{pp}\) donde
\(\lambda\) es un valor escalar, y \(\mathbf{Y}_r\) el vector columna de
respuestas con \(p\) ceros añadidos, en ambos casos al final de la
matriz o vector.

Recordando que el estimador por regresión ridge se define de la
siguiente manera:

\[
\hat{\beta}_R = \left(\mathbf{X}^T\mathbf{X} + \lambda \mathbf{I}\right)^{-1}\mathbf{X}^T\mathbf{Y}\tag{1}
\]

Y que el estimador de mínimos cuadrados tiene la siguiente forma:

\[
\hat{\beta} = \left(\mathbf{X}^T\mathbf{X}\right)^{-1}\mathbf{X}^T\mathbf{Y}\tag{2}
\]

Entonces, considerando las matrices \(\mathbf{X}_r\) y \(\mathbf{Y}_r\)
y aplicando la regresión de mínimos cuadrados, se tiene lo siguiente:

\[
\hat{\beta}_r = \left(\mathbf{X}_r^T\mathbf{X}_r\right)^{-1}\mathbf{X}_r^T\mathbf{Y}_r\tag{3}
\]

Si se desglosan las matrices de la siguiente manera:

\[
\mathbf{X}_r &= \begin{bmatrix}\mathbf{X} \\ \sqrt{\lambda}\mathbf{I}_{pp}\end{bmatrix}; &
\mathbf{Y}_r &= \begin{bmatrix}\mathbf{Y} \\ \mathbf{0}_{p}\end{bmatrix}
\] Vale la pena recalcar que la notación \(\mathbf{0}_{p}\) indica un
vector columna de ceros de tamaño p.~Sustituyendo en la ecuación \(3\):
\$\$\begin{align*}
\hat{\beta}_r &= \left(
    \begin{bmatrix}\mathbf{X}^T & \sqrt{\lambda}\mathbf{I}_{pp}\end{bmatrix}
    \begin{bmatrix}\mathbf{X} \\ \sqrt{\lambda}\mathbf{I}_{pp}\end{bmatrix}
    \right)^{-1}
    \begin{bmatrix}\mathbf{X}^T & \sqrt{\lambda}\mathbf{I}_{pp}\end{bmatrix}
    \begin{bmatrix}\mathbf{Y} \\ \mathbf{0}_{p}\end{bmatrix} \\
    
    &=
    \left(\mathbf{X}^T\mathbf{X} + \sqrt{\lambda}\mathbf{I}_{pp}\sqrt{\lambda}\mathbf{I}_{pp}\right)^{-1}
    \begin{bmatrix}\mathbf{X}^T & \sqrt{\lambda}\mathbf{I}_{pp}\end{bmatrix}
    \begin{bmatrix}\mathbf{Y} \\ \mathbf{0}_{p}\end{bmatrix} \\

    &=
    \left(\mathbf{X}^T\mathbf{X} + \sqrt{\lambda}^2\mathbf{I}_{pp}\right)^{-1}
    \begin{bmatrix}\mathbf{X}^T & \sqrt{\lambda}\mathbf{I}_{pp}^2\end{bmatrix}
    \begin{bmatrix}\mathbf{Y} \\ \mathbf{0}_{p}\end{bmatrix} \\

    &=
    \left(\mathbf{X}^T\mathbf{X} + \lambda\mathbf{I}_{pp}\right)^{-1}
    \left(\mathbf{X}^T \mathbf{Y} + \sqrt{\lambda}\mathbf{I}_{pp}\mathbf{0}_{p}\right) \\

    \hat{\beta}_r &=
    \left(\mathbf{X}^T\mathbf{X} + \lambda \mathbf{I}_{pp}\right)^{-1}\mathbf{X}^T\mathbf{Y}
\end{align*}\[
\]\therefore \hat{\beta}\_r = \hat{\beta}\_R\$\$

Vale la pena recordarse que la matriz identidad es idempotente, por lo
que puede hacerse el segundo paso mostrado. Con esto, puede observarse
que el resultado obtenido es idéntico al mostrado en la ecuación \(1\).
Con esto concluye la demostración. \(\blacksquare\)

    \hypertarget{ejercicio-2}{%
\subsection{Ejercicio 2}\label{ejercicio-2}}

Los valores ajustados \(\widehat{\mathbf{Y}}\) por mínimos cuadrados
usuales satisfacen la igualdad \(\widehat{\mathbf{Y}} = \mathbf{PY}\)
donde \(\mathbf{PY}\) es la matriz de proyección ortogonal. Una
consecuencia de este hecho es que los residuales
\(\widehat{\epsilon} =\widehat{\mathbf{Y}} - \mathbf{Y}\) y el vector de
respuestas \(Y\) son ortogonales. 1. Demuestre que la matriz
\(\mathbf{P}_\lambda = \mathbf{X}\left(\mathbf{X}^T\mathbf{X} + \lambda \mathbf{I}\right)^{-1}\mathbf{X}^T\),
asociada a la regresión ridge, no es una matriz de proyección para todo
\(\lambda > 0\) 2. Demuestre que para cualquier \(\lambda > 0\) los
valores ajustados \(\widehat{\mathbf{Y}}(\lambda)\) no son ortogonales
con los residuales ridge
\(\widehat{\epsilon} =\widehat{\mathbf{Y}}(\lambda) - \mathbf{Y}\).

    \hypertarget{respuesta}{%
\subsubsection{Respuesta}\label{respuesta}}

Para el primer inciso, se sabe que una matriz de proyección debe ser
idempotente. Como es una condición necesaria para ser una matriz de
proyección, con que no cumpla esta condición para todo \(\lambda>0\)
basta para demostrar que no es una matriz de proyección.

En primera instancia, considerando la descomposición SVD de
\(\mathbf{X} = UDV^T\), se obtiene el valor de \(\mathbf{P}_\lambda\):

Considerando la descomposición SVD de \(\mathbf{X} = UDV^T\), se obtiene
lo siguiente: \$\$\begin{align*}
\mathbf{P}_\lambda &= 
UDV^T\left(VD^TU^TUDV^T + \lambda \mathbf{I}\right)^{-1}VD^TU^T \\

&= 
UDV^T\left(VD^TDV^T + \lambda \mathbf{I}\right)^{-1}VD^TU^T\\

&= 
UDV^T\left(VD^TDV^T + \lambda VV^T\right)^{-1}VD^TU^T \\

&= 
UDV^T\left(D^TDV^T + \lambda V^T\right)^{-1}(V)^{-1}VD^TU^T \\

&=
UDV^T(V^T)^{-1}\left(D^TD + \lambda \mathbf{I}\right)^{-1}(V)^{-1}VD^TU^T \\ 

&=
UD\left(D^TD + \lambda \mathbf{I}\right)^{-1}D^TU^T\tag{4}
\end{align*}\$\$

Si esta matriz fuera idempotente, entonces su traza sería equivalente a
su rango. Esta demostración se omite, pero puede obtenerse relativamente
fácil de encontrar los eigenvalores de una matriz idempotente. Debido a
que debe cumplirse para una matriz idempotente:
\[\mathrm{tr}(\mathbf{P}) = \mathrm{rank}(\mathbf{P})\] Puede usarse
como criterio para determinar si \(\mathbf{P}_\lambda\) es una matriz de
proyección. Obteniendo la traza del resultado mostrado en \(4\):
\$\$\begin{align*}
\mathrm{tr}(\mathbf{P}_\lambda) &= \mathrm{tr}\left(UD\left(D^TD + \lambda \mathbf{I}\right)^{-1}D^TU^T\right) \\

&= 
\mathrm{tr}\left(D\left(D^TD + \lambda \mathbf{I}\right)^{-1}D^T\right)
\end{align*}\[
Debido a que $D$ es una matriz diagonal, entonces puede obtenerse la traza sumando tal cual cada uno de los valores de la siguiente manera:
\]\mathrm{tr}(\mathbf{P}\emph{\lambda) = \sum}\{i=1\}\^{}p
\frac{d_i^2}{d_i^2+\lambda}\[
Donde $i$ representa cada uno de los valores singulares. Puede observarse que 
para todo $\lambda>0$ el $i$-ésimo elemento de esta suma es menor que 1, con la siguiente desigualdad:
\]\frac{d_i^2}{d_i^2+\lambda} \textless{} 1;\quad d\_i\^{}2 \textless{}
d\_i\^{}2 + \lambda\[
Debido a que la suma de cada uno de estos elementos es menor que 1, si se suman todos los elementos:
\]\mathrm{tr}(\mathbf{P}\_\lambda)\textless{}\mathrm{rank}(D)\$\$ Esto
puede estarse seguro ya que el número de valores singulares es igual al
rango de la matriz \(D\). Con esto concluye la demostración del primer
inciso. \(\blacksquare\)

Para el segundo inciso, conviene recordar que si un par de vectores son
ortogonales entonces el producto punto (o la proyección,
alternativamente) en otro es cero. Alternativamente, sean \(v\) y \(w\)
dos vectores de la misma dimensión que además son ortogonales, entonces
se cumple la siguiente igualdad: \[u\cdot v = u^Tv = v^Tu = 0\] Donde
\((\cdot)\) representa el producto punto. Entonces, sean los residuales
de ridge \(\mathbf{\widehat{Y}-Y} = \widehat{\epsilon}\), y
\(\mathbf{\widehat{Y} = \mathbf{P}_\lambda Y}\). Entonces, la proyección
de un vector en el otro es dado por la siguiente ecuación: \$\$
\begin{gather*}
    (\mathbf{P}_\lambda \mathbf{Y})^T \left(\mathbf{\widehat{Y}-Y}\right) \\

    \mathbf{Y}^T\mathbf{P}_\lambda^T \left(\mathbf{\mathbf{P}_\lambda \mathbf{Y}-Y}\right) \\

    \mathbf{Y}^T\mathbf{P}_\lambda^T \left(\mathbf{\mathbf{P}_\lambda -I}\right)\mathbf{Y} \\

    \mathbf{Y}^T \left(\mathbf{P}_\lambda^T\mathbf{\mathbf{P}_\lambda -\mathbf{P}_\lambda^T}\right)\mathbf{Y}
    \end{gather*} \[
Recordando que $\mathbf{P}_\lambda = \mathbf{X}\left(\mathbf{X}^T\mathbf{X} + \lambda \mathbf{I}\right)^{-1}\mathbf{X}^T$, y sea $\mathbf{A} = \left(\mathbf{X}^T\mathbf{X} + \lambda \mathbf{I}\right)^{-1}$, entonces:
\]\mathbf{P}\_\lambda = \mathbf{X}\mathbf{A}\mathbf{X}\^{}T\[
\] \begin{gather*}
\mathbf{Y}^T \left((\mathbf{X}\mathbf{A}\mathbf{X}^T)^T\mathbf{X}\mathbf{A}\mathbf{X}^T -\mathbf{X}\mathbf{A}\mathbf{X}^T\right)\mathbf{Y} \\

\mathbf{Y}^T \left(\mathbf{X}\mathbf{A}^T\mathbf{X}^T\mathbf{X}\mathbf{A}\mathbf{X}^T -\mathbf{X}\mathbf{A}\mathbf{X}^T\right)\mathbf{Y}\tag{5}
\end{gather*} \[
Transponiendo $\mathbf{A}$:
\]\mathbf{A}\^{}T = \left(\left(\mathbf{X}\^{}T\mathbf{X} +
\lambda \mathbf{I}\right)\textsuperscript{\{-1\}\right)}T\[
Sabiendo que es invertible:
\]\begin{align*}
\mathbf{A}^T &= \left(\left(\mathbf{X}^T\mathbf{X} + \lambda \mathbf{I}\right)^T\right){-1} \\
\mathbf{A}^T &= \left(\left((\mathbf{X}^T\mathbf{X})^T + (\lambda \mathbf{I})?T\right)\right){-1} \\
\mathbf{A}^T &= \left(\left(\mathbf{X}^T\mathbf{X} + \lambda \mathbf{I}\right)\right){-1} \\
\mathbf{A}^T &= \mathbf{A}
\end{align*}\$\$

Por lo que, sustituyendo en el resultado en \(5\):
\[\mathbf{Y}^T \left(\mathbf{X}\mathbf{A}\mathbf{X}^T\mathbf{X}\mathbf{A}\mathbf{X}^T -\mathbf{X}\mathbf{A}\mathbf{X}^T\right)\mathbf{Y}\]
\[\mathbf{Y}^T \left(\mathbf{P^2}_\lambda -\mathbf{P}_\lambda\right)\mathbf{Y}\]

En el inciso anterior se demostró que la matriz \(\mathbf{P}_\lambda\)
no es idempotente, por lo que el término
\(\mathbf{P^2}_\lambda -\mathbf{P}_\lambda\) es distinto a
\(\mathbf{0}\). Entonces, no se cumple la ortogonalidad de las
predicciones contra los residuales estimados de Ridge, ya que este
resultado es distinto de cero para todo \(\mathbf{Y}\) que sea distinto
al vector 0, y con esto concluye la demostración.\(\blacksquare\)

Vale la pena mencionarse que en el ejercicio 1 de la tarea se ve que
puede hacerse una expresión de la regresión Ridge equivalente a mínimos
cuadrados, donde efectivamente se cumple la ortogonalidad de la matriz.
Sin embargo, debido a que se incluyeron filas adicionales tanto en las
variables regresoras como en la de respuesta, corresponderían a
observaciones que no existieron, por lo que no tiene por qué cumplirse
la ortogonalidad con los datos reales (es decir, aquellos que no se les
``agregaron'' otros valores).

    \hypertarget{ejercicio-3}{%
\subsection{Ejercicio 3}\label{ejercicio-3}}

Algunas veces se tiende a pensar que se puede detectar una matriz mal
condicionada al observar valores pequeños en la diagonal de la matriz. Y
que, por lo tanto, una matriz sin valores pequeños en la diagonal estará
bien condicionada. Sin embargo, esto no es cierto. Para ilustrar este
hecho, considere las matrices:

\[
\begin{bmatrix}
0.501 & -1    & 0      & ...   & 0  \\
0     & 0.502 & -1     &       & 0  \\
\vdots&       & \ddots & \ddots &  \\
0     & 0     & 0      & 0.599 & -1 \\
0     & 0     & 0      & 0     & 0.600
\end{bmatrix} 
\quad \mathrm{y} \quad 
\begin{bmatrix}
1 & -1 & -1     & ...  & -1  \\
0 & 1  & -1     &      & -1 \\
\vdots &   & \ddots & \ddots  & \\
0 & 0  & 0      & 1    & -1 \\
0 & 0  & 0      & 0    & 1
\end{bmatrix}
\]

Halle el número de condición de dichas matrices, usando este número de
condición se puede observar que ambas matrices están mal condicionadas,
aunque ninguna tiene elementos pequeños en la diagonal.

    \hypertarget{respuesta}{%
\subsubsection{Respuesta}\label{respuesta}}

Sea \(A\) la primera matriz y \(B\) la segunda matriz mostrada. Debido a
que no se especifica la dimensión de B, se asume un tamaño arbitrario.
Para la matriz \(A\) su tamaño sí es determinado por lo que puede
calcularse directamente el número de condición, con el siguiente script
\footnote{No todos los scripts de Python mostrados en esta tarea se
  escriben para ser corridos individualmente y encontrarse encapsulados.
  Si se corren secuencialmente, entonces no deben existir errores. Se
  hace de esta manera para que cuando se refiera a un ejemplo, pueda
  describirse inmediatamente al código al que se refiere.} de Python:

    \begin{tcolorbox}[breakable, size=fbox, boxrule=1pt, pad at break*=1mm,colback=cellbackground, colframe=cellborder]
\prompt{In}{incolor}{2}{\boxspacing}
\begin{Verbatim}[commandchars=\\\{\}]
\PY{n}{A} \PY{o}{=} \PY{n}{np}\PY{o}{.}\PY{n}{diag}\PY{p}{(}\PY{n}{np}\PY{o}{.}\PY{n}{arange}\PY{p}{(}\PY{l+m+mf}{0.501}\PY{p}{,}\PY{l+m+mf}{0.6001}\PY{p}{,}\PY{l+m+mf}{0.001}\PY{p}{)}\PY{p}{)} \PY{o}{+} \PY{n}{np}\PY{o}{.}\PY{n}{diag}\PY{p}{(}\PY{o}{\PYZhy{}}\PY{n}{np}\PY{o}{.}\PY{n}{ones}\PY{p}{(}\PY{l+m+mi}{99}\PY{p}{)}\PY{p}{,}\PY{l+m+mi}{1}\PY{p}{)}
\PY{n}{cond\PYZus{}A} \PY{o}{=} \PY{n}{np}\PY{o}{.}\PY{n}{linalg}\PY{o}{.}\PY{n}{cond}\PY{p}{(}\PY{n}{A}\PY{p}{)}
\PY{n+nb}{print}\PY{p}{(}\PY{l+s+sa}{f}\PY{l+s+s1}{\PYZsq{}}\PY{l+s+s1}{Número de condición de la matriz A: }\PY{l+s+si}{\PYZob{}}\PY{n}{cond\PYZus{}A}\PY{l+s+si}{:}\PY{l+s+s1}{.4e}\PY{l+s+si}{\PYZcb{}}\PY{l+s+s1}{\PYZsq{}}\PY{p}{)}
\end{Verbatim}
\end{tcolorbox}

    \begin{Verbatim}[commandchars=\\\{\}]
Número de condición de la matriz A: 2.2097e+26
    \end{Verbatim}

    Para la matriz B se propone graficar el número de condición contra la
dimensión de la matriz.

    \begin{tcolorbox}[breakable, size=fbox, boxrule=1pt, pad at break*=1mm,colback=cellbackground, colframe=cellborder]
\prompt{In}{incolor}{3}{\boxspacing}
\begin{Verbatim}[commandchars=\\\{\}]
\PY{n}{maxdim} \PY{o}{=} \PY{l+m+mi}{50}
\PY{n}{ncond} \PY{o}{=} \PY{p}{[}\PY{p}{]}
\PY{k}{for} \PY{n}{n} \PY{o+ow}{in} \PY{n+nb}{range}\PY{p}{(}\PY{l+m+mi}{1}\PY{p}{,}\PY{n}{maxdim}\PY{p}{)}\PY{p}{:}
    \PY{n}{ones} \PY{o}{=} \PY{n}{np}\PY{o}{.}\PY{n}{ones}\PY{p}{(}\PY{n}{n}\PY{p}{)}
    \PY{n}{B} \PY{o}{=} \PY{n}{np}\PY{o}{.}\PY{n}{triu}\PY{p}{(}\PY{o}{\PYZhy{}}\PY{n}{ones}\PY{p}{,}\PY{l+m+mi}{1}\PY{p}{)}\PY{o}{+}\PY{n}{np}\PY{o}{.}\PY{n}{diag}\PY{p}{(}\PY{n}{ones}\PY{p}{)}
    \PY{n}{ncond}\PY{o}{.}\PY{n}{append}\PY{p}{(}\PY{n}{np}\PY{o}{.}\PY{n}{linalg}\PY{o}{.}\PY{n}{cond}\PY{p}{(}\PY{n}{B}\PY{p}{)}\PY{p}{)}

\PY{n}{plt}\PY{o}{.}\PY{n}{yscale}\PY{p}{(}\PY{l+s+s1}{\PYZsq{}}\PY{l+s+s1}{log}\PY{l+s+s1}{\PYZsq{}}\PY{p}{)}
\PY{n}{plt}\PY{o}{.}\PY{n}{grid}\PY{p}{(}\PY{p}{)}
\PY{n}{plt}\PY{o}{.}\PY{n}{grid}\PY{p}{(}\PY{n}{which}\PY{o}{=}\PY{l+s+s1}{\PYZsq{}}\PY{l+s+s1}{minor}\PY{l+s+s1}{\PYZsq{}}\PY{p}{,}\PY{n}{alpha}\PY{o}{=}\PY{l+m+mf}{0.2}\PY{p}{)}
\PY{n}{plt}\PY{o}{.}\PY{n}{plot}\PY{p}{(}\PY{n}{ncond}\PY{p}{)}
\PY{n}{plt}\PY{o}{.}\PY{n}{xlabel}\PY{p}{(}\PY{l+s+s1}{\PYZsq{}}\PY{l+s+s1}{Dimensión de la matriz}\PY{l+s+s1}{\PYZsq{}}\PY{p}{)}
\PY{n}{plt}\PY{o}{.}\PY{n}{ylabel}\PY{p}{(}\PY{l+s+s1}{\PYZsq{}}\PY{l+s+s1}{Número de condición de la matriz}\PY{l+s+s1}{\PYZsq{}}\PY{p}{)}
\PY{n}{plt}\PY{o}{.}\PY{n}{title}\PY{p}{(}\PY{l+s+s1}{\PYZsq{}}\PY{l+s+s1}{Número de condición de la matriz B contra dimensión}\PY{l+s+s1}{\PYZsq{}}\PY{p}{)}
\end{Verbatim}
\end{tcolorbox}

            \begin{tcolorbox}[breakable, size=fbox, boxrule=.5pt, pad at break*=1mm, opacityfill=0]
\prompt{Out}{outcolor}{3}{\boxspacing}
\begin{Verbatim}[commandchars=\\\{\}]
Text(0.5, 1.0, 'Número de condición de la matriz B contra dimensión')
\end{Verbatim}
\end{tcolorbox}
        
    \begin{center}
    \adjustimage{max size={0.9\linewidth}{0.9\paperheight}}{Tarea1_JLAC_files/Tarea1_JLAC_10_1.png}
    \end{center}
    { \hspace*{\fill} \\}
    
    Puede verse que el número de condición de la matriz \(B\) crece
rápidamente contra la dimensión de la matriz.

    \hypertarget{ejercicio-4}{%
\subsection{Ejercicio 4}\label{ejercicio-4}}

Considere la matriz
\(\mathbf{A} = \begin{bmatrix}1 & \alpha \\ \alpha & 1\end{bmatrix}\) ,
halle su número de condición, \(\kappa(\mathbf{A})\), y demuestre que
\(\lim_{\alpha\rightarrow1}\kappa(\mathbf{A}) = \infty\). Por otro lado,
considere la matriz
\(\mathbf{B} = \begin{bmatrix} \alpha & 0\\ 0& \alpha\end{bmatrix}\), se
podría pensar que cuando \(\alpha\rightarrow0\), la matriz B está mal
condicionada. Halle el número de condición de la matriz \(\mathbf{B}\),
\(\kappa(\mathbf{B})\), y calcule
\(\lim_{\alpha\rightarrow0}\kappa(\mathbf{B})\), ¿es cierto que cuando
\(\alpha\rightarrow0\), \(\kappa(\mathbf{B})\) se vuelve una matriz mal
condicionada?

    \hypertarget{respuesta}{%
\subsubsection{Respuesta}\label{respuesta}}

Para encontrar el número de condición de \(\mathbf{A}\), se aprovecha
que es una matriz normal. Para demostrar que es normal, basta con ver
que \(\mathbf{A}= \mathbf{A}^T\), por lo que
\(\mathbf{A}\mathbf{A}^T = \mathbf{A}^T\mathbf{A}\). Debido a que es una
matriz normal con entradas reales, su número de condición puede
calcularse de la siguiente manera:
\[\kappa(\mathbf{A}) = \left|\frac{\lambda_{\max}(\mathbf{A})}{\lambda_{\min}(\mathbf{A})}\right|\]

Entonces, calculando los eigenvalores de la matriz \(\mathbf{A}\)
resolviendo su polinomio característico: \[\begin{align*}
(1-\lambda)^2-\alpha^2 &= 0 \\
(1-\lambda+\alpha)(1-\lambda-\alpha) &= 0
\end{align*}\] Debido a que solamente hay dos eigenvalores, el mayor en
magnitud corresponde al máximo, y el menor al mínimo. Por lo tanto:
\[\lambda = \left\{1+\alpha,1-\alpha\right\}\] Para demostrar que el
límite tiende al infinito, entonces para todo número \(N>1\)\footnote{Se
  elige mayor que 1 para evitar indeterminaciones en el caso que
  \(\alpha = 1+\delta\).} existe un \(\delta>0\) tal que para todo
\(\alpha\):
\[0<|\alpha-1|<\delta \implies \kappa(\mathbf{A}) = \left|\frac{\lambda_{\max}(\mathbf{A})}{\lambda_{\min}(\mathbf{A})}\right| > N \]
Debido a que \(\alpha\rightarrow1\), \(1+\alpha>1-\alpha\). Entonces,
sea \(\alpha = 1+\delta\): \[\begin{align*}
\left|\frac{1+\alpha}{1-\alpha}\right| &> N \\
\left|\frac{1+1+\delta}{1-1-\delta}\right| &> N \\
\left|\frac{2+\delta}{-\delta}\right| &> N \\
\frac{2}{\delta}+1 &>N \\
\frac{2}{N-1} &> \delta
\end{align*}\] El caso en que \(\alpha = 1-\delta\): \[\begin{align*}
\left|\frac{1+\alpha}{1-\alpha}\right| &> N \\
\left|\frac{1+1-\delta}{1-1+\delta}\right| &> N \\
\left|\frac{2-\delta}{\delta}\right| &> N \\
\frac{2}{\delta}-1 &>N \\
\frac{2}{N+1} &> \delta \\
\frac{2}{N-1} &> \frac{2}{N+1}
\end{align*}\]

Por lo que con elegir un \(\delta \in (0,\frac{2}{N+1})\) se cumple la
desigualdad, y se demuestra que tiende al infinito. \(\blacksquare\)

En el caso de la matriz \(\mathbf{B}\), se sigue un procedimiento muy
similar. \(\mathbf{B} = \alpha\mathbb{I}\), por lo que sus eigenvalores
serán \(\alpha\) en ambos casos. De cualquier forma, calculando su
polinomio característico:
\[(\alpha-\lambda)^2 = 0\implies \lambda = \alpha\] Entonces, para todo
\(\alpha \ne 0\), debido a que ambos eigenvalores son iguales, y
\(\mathbf{B}\) es una matriz normal, entonces se tiene que:
\[\kappa(\mathbf{B}) = \frac{\alpha}{\alpha} = 1;\quad {\alpha\ne0}\]
Entonces, \(\lim_{\alpha\rightarrow0}\kappa(\mathbf{B}) =1\). La
demostración es trivial por lo que no se realiza --con elegir cualquier
\(\delta>0\) se satisface que \(1-\kappa(\mathbf{B})<\epsilon\).

    \hypertarget{ejercicio-5}{%
\subsection{Ejercicio 5}\label{ejercicio-5}}

Sea \(X\) un vector aleatorio \(n\times1\) y sea \(\mathbb{A}\) una
matriz \(n\times n\). Si \(\mathbb{E}(X) = \mu\) y
\(\mathbb{V}(X) = \Sigma\), demuestre que:
\[\mathbb{E}(X^T\mathbf{A}X) = \mathrm{tr}(\mathbf{A}\Sigma)+\mu^T\mathbf{A}\mu\]

    \hypertarget{respuesta}{%
\subsubsection{Respuesta}\label{respuesta}}

\[\mathbb{E}(X^T\mathbf{A}X)\] Por linealidad de una transformación
lineal:
\[X^T\mathbf{A}X = \left(X-\mu\right)^T\mathbf{A}\left(X-\mu\right) + \mu^T\mathbf{A}\mu\]
Por linealidad del valor esperado: \[\begin{align*}
\mathbb{E}(\left(X-\mu\right)^T\mathbf{A}\left(X-\mu\right) + \mu^T\mathbf{A}\mu) &= \mathbb{E}(\left(X-\mu\right)^T\mathbf{A}\left(X-\mu\right))+ \mathbb{E}(\mu^T\mathbf{A}\mu) \\
&= \mathbb{E}(\left(X-\mu\right)^T\mathbf{A}\left(X-\mu\right)) + \mu^T\mathbf{A}\mu
\end{align*}\]

Calculando la traza de la expresión:
\[\mathrm{tr}(\mathbb{E}(X^T\mathbf{A}X)) = \mathbb{E}(X^T\mathbf{A}X) = \mathrm{tr}(\mathbb{E}(\left(X-\mu\right)^T\mathbf{A}\left(X-\mu\right)) + \mu^T\mathbf{A}\mu)\]
Por linealidad de la traza: \[\begin{align*}
\mathbb{E}(X^T\mathbf{A}X) &= \mathrm{tr}(\mathbb{E}(\left(X-\mu\right)^T\mathbf{A}\left(X-\mu\right))) + \mathrm{tr}(\mu^T\mathbf{A}\mu) \\
&= \mathrm{tr}(\mathbb{E}(\left(X-\mu\right)^T\mathbf{A}\left(X-\mu\right))) + \mu^T\mathbf{A}\mu\tag{5}
\end{align*}\] Por la propiedad cíclica de la traza:
\[\mathrm{tr}(\mathbb{E}(\left(X-\mu\right)^T\mathbf{A}\left(X-\mu\right))) = \mathrm{tr}(\mathbb{E}(\mathbf{A}\left(X-\mu\right)\left(X-\mu\right)^T))\]
Por linealidad del valor esperado:
\[\mathrm{tr}(\mathbb{E}(\mathbf{A}\left(X-\mu\right)\left(X-\mu\right)^T)) = \mathrm{tr}(\mathbf{A}\mathbb{E}(\left(X-\mu\right)\left(X-\mu\right)^T))\]
Recordando la definición de la covarianza:
\[\mathbb{V}(X) = \Sigma = \mathbb{E}(\left(X-\mu\right)\left(X-\mu\right)^T)\]
Sustituyendo:
\[\mathrm{tr}(\mathbb{E}(\mathbf{A}\left(X-\mu\right)\left(X-\mu\right)^T)) = \mathrm{tr}(A\Sigma)\]
Y sustituyendo nuevamente en \(5\):
\[\mathbb{E}(X^T\mathbf{A}X) = \mathrm{tr}(A\Sigma) + \mu^T\mathbf{A}\mu\]
Y con esto se termina la demostración. \(\blacksquare\)

    \hypertarget{ejercicio-6}{%
\subsection{Ejercicio 6}\label{ejercicio-6}}

La siguiente tabla muestra datos relacionados con las necesidades para
el trabajo hospitalario, provenientes de 17 hospitales. La variable
\(V_1\) representa la carga promedio diaria de pacientes, \(V_2\) denota
el número mensual de servicios de rayos X, \(V_3\) denota el número de
días-cama de ocupación mensual, \(V_4\) denota la población elegible en
el área (dividida por 1000), \(V_5\) denota la longitud promedio de
permanencia de los pacientes, en días. La variable respuesta \(W\) es el
número de horas de trabajo durante el mes.

Sea \(\mathbf{X}\) la matriz formada por las variables
\(V_1,\ V_2,\ V_3,\ V_4\) y \(V_5\). Haga lo siguiente: \#\#\#\#
Análisis de colinealidad 1. Realice los gráficos de dispersión de la
variable \(W\) con cada una de las variables de \(\mathbf{X}\), así como
los gráficos de dispersión entre las variables de \(\mathbf{X}\).
También evalúe los coeficientes de correlación entre la variable
respuesta y cada una de las cinco variables regresoras. 2. Calcule el
número de condición de la matriz \(\mathbf{X}\) y su matriz de
correlación. ¿Qué evidencia hay de la presencia de multicolinealidad en
los datos de este problema? 3. ¿Cómo cambia el número de condición al
centrar y escalar la matriz \(\mathbf{X}\)? Sea \(\mathbf{X}_c\) la
matriz \(\mathbf{X}\) centrada y escalada. 4. Calcule e interprete los
factores de inflación de la varianza (VIF) para las cinco variables
regresoras, considerando la matriz \(\mathbf{X}_c\). 5. Calcule los
índices de condición de la matriz \(\mathbf{X}_c\), realice la
descomposición de la varianza de los parámetros de regresión. Presente
sus resultados en la llamada matriz \(\Pi\) e interprételos. \#\#\#\#
Regresión ridge 1. Una manera de estimar el valor óptimo del parámetro
ridge es la siguiente:
\[\hat{\lambda} = p\frac{\hat{\sigma}^2}{\hat{\beta}^T\hat{\beta}}\]
Donde \(\hat{\beta}\) son los parámetros de regresión estimados por
mínimos cuadrados ordinarios y \(\hat{\sigma}^2\) es un estimador de la
varianza de los errores independientes del modelo; por ejemplo
\(\hat{\sigma}^2 = \frac{1}{n-p}SSR\), donde \(SSR\) es la suma de
cuadrados de residuales. Calcule este estimador del parámetro ridge.

\begin{enumerate}
\def\labelenumi{\arabic{enumi}.}
\setcounter{enumi}{1}
\tightlist
\item
  Determine el valor del parámetro ridge utilizando el método de
  validación cruzada.
\item
  Estime para ambos valores de \(\lambda\), obtenidos en los incisos
  anteriores, los coeficientes de regresión ridge \(\hat{\beta}_R\), así
  como los valores ajustados de la variable respuesta
  \(\hat{\mathbf{Y}}\).
\item
  Presente en dos gráficos los puntos \((Y_i,\hat{Y}_i)\) para los dos
  valores de \(\lambda\), agregue en las gráficas la línea identidad.
  Interprete la gráfica.
\item
  Grafique la traza de ridge y dibuje con dos líneas vertiales los
  valores estimados de \(\lambda\).
\item
  Los factores de inflación de varianza usuales se pueden definir como:
  \[VIF_j=n(\mathbf{X}_c^T\mathbf{X}_c)^{-1}_{jj},j=\{1,2,...,p\}\]
  Similarmente, los factores de inflación de varianza ridge se pueden
  definir como:
  \[VIF_j=n\left[(\mathbf{X}_c^T\mathbf{X}_c+\lambda I)^{-1}\mathbf{X}_c^T\mathbf{X}_c(\mathbf{X}_c^T\mathbf{X}_c+\lambda I)^{-1}\right]_{jj}\]
  Calcule estos factores de inflación de varianza.
\item
  Comente sus resultados.
\end{enumerate}

    \hypertarget{respuestas}{%
\subsubsection{Respuestas}\label{respuestas}}

Debido a la naturaleza del ejercicio 6, se realizarán los fragmentos de
código relevantes en cada una de las respuestas. Se reconoce que si se
corren individualmente los segmentos no necesariamente serán correctos;
sin embargo, si se corren secuencialmente, se asegura que todos
funcionen. A continuación se cargan los datos.

    \begin{tcolorbox}[breakable, size=fbox, boxrule=1pt, pad at break*=1mm,colback=cellbackground, colframe=cellborder]
\prompt{In}{incolor}{4}{\boxspacing}
\begin{Verbatim}[commandchars=\\\{\}]
\PY{c+c1}{\PYZsh{} Datos}
\PY{n}{df} \PY{o}{=} \PY{n}{pd}\PY{o}{.}\PY{n}{DataFrame}\PY{p}{(}\PY{n}{data}\PY{o}{=} 
    \PY{p}{\PYZob{}}
        \PY{l+s+s1}{\PYZsq{}}\PY{l+s+s1}{V1}\PY{l+s+s1}{\PYZsq{}}\PY{p}{:} \PY{p}{[}\PY{l+m+mf}{15.57}\PY{p}{,}\PY{l+m+mf}{44.02}\PY{p}{,}\PY{l+m+mf}{20.42}\PY{p}{,}\PY{l+m+mf}{18.74}\PY{p}{,}\PY{l+m+mf}{49.2}\PY{p}{,}\PY{l+m+mf}{44.92}\PY{p}{,}\PY{l+m+mf}{55.48}\PY{p}{,}\PY{l+m+mf}{59.28}\PY{p}{,}\PY{l+m+mf}{94.39}\PY{p}{,}\PY{l+m+mf}{128.02}\PY{p}{,}\PY{l+m+mi}{96}\PY{p}{,}\PY{l+m+mf}{131.42}\PY{p}{,}\PY{l+m+mf}{127.21}\PY{p}{,}\PY{l+m+mf}{252.9}\PY{p}{,}\PY{l+m+mf}{409.2}\PY{p}{,}\PY{l+m+mf}{463.7}\PY{p}{,}\PY{l+m+mf}{510.22}\PY{p}{]}\PY{p}{,}
        \PY{l+s+s1}{\PYZsq{}}\PY{l+s+s1}{V2}\PY{l+s+s1}{\PYZsq{}}\PY{p}{:} \PY{p}{[}\PY{l+m+mi}{2463}\PY{p}{,}\PY{l+m+mi}{2048}\PY{p}{,}\PY{l+m+mi}{3940}\PY{p}{,}\PY{l+m+mi}{6505}\PY{p}{,}\PY{l+m+mi}{5723}\PY{p}{,}\PY{l+m+mi}{11520}\PY{p}{,}\PY{l+m+mi}{5779}\PY{p}{,}\PY{l+m+mi}{5969}\PY{p}{,}\PY{l+m+mi}{8461}\PY{p}{,}\PY{l+m+mi}{20106}\PY{p}{,}\PY{l+m+mi}{13313}\PY{p}{,}\PY{l+m+mi}{10771}\PY{p}{,}\PY{l+m+mi}{15543}\PY{p}{,}\PY{l+m+mi}{36194}\PY{p}{,}\PY{l+m+mi}{34703}\PY{p}{,}\PY{l+m+mi}{39204}\PY{p}{,}\PY{l+m+mi}{86533}\PY{p}{]}\PY{p}{,}
        \PY{l+s+s1}{\PYZsq{}}\PY{l+s+s1}{V3}\PY{l+s+s1}{\PYZsq{}}\PY{p}{:} \PY{p}{[}\PY{l+m+mf}{472.82}\PY{p}{,}\PY{l+m+mf}{1339.75}\PY{p}{,}\PY{l+m+mf}{620.25}\PY{p}{,}\PY{l+m+mf}{568.33}\PY{p}{,}\PY{l+m+mf}{1497.6}\PY{p}{,}\PY{l+m+mf}{1365.83}\PY{p}{,}\PY{l+m+mi}{1687}\PY{p}{,}\PY{l+m+mf}{1639.92}\PY{p}{,}\PY{l+m+mf}{2872.33}\PY{p}{,}\PY{l+m+mf}{3655.08}\PY{p}{,}\PY{l+m+mi}{2912}\PY{p}{,}\PY{l+m+mi}{3921}\PY{p}{,}\PY{l+m+mf}{3865.67}\PY{p}{,}\PY{l+m+mf}{7684.1}\PY{p}{,}\PY{l+m+mf}{12246.33}\PY{p}{,}\PY{l+m+mf}{14098.4}\PY{p}{,}\PY{l+m+mi}{15524}\PY{p}{]}\PY{p}{,}
        \PY{l+s+s1}{\PYZsq{}}\PY{l+s+s1}{V4}\PY{l+s+s1}{\PYZsq{}}\PY{p}{:} \PY{p}{[}\PY{l+m+mi}{18}\PY{p}{,}\PY{l+m+mf}{9.5}\PY{p}{,}\PY{l+m+mf}{12.8}\PY{p}{,}\PY{l+m+mf}{36.7}\PY{p}{,}\PY{l+m+mf}{35.7}\PY{p}{,}\PY{l+m+mi}{24}\PY{p}{,}\PY{l+m+mf}{43.3}\PY{p}{,}\PY{l+m+mf}{46.7}\PY{p}{,}\PY{l+m+mf}{78.7}\PY{p}{,}\PY{l+m+mf}{180.5}\PY{p}{,}\PY{l+m+mf}{60.9}\PY{p}{,}\PY{l+m+mf}{103.7}\PY{p}{,}\PY{l+m+mf}{126.8}\PY{p}{,}\PY{l+m+mf}{157.7}\PY{p}{,}\PY{l+m+mf}{169.4}\PY{p}{,}\PY{l+m+mf}{331.4}\PY{p}{,}\PY{l+m+mf}{371.6}\PY{p}{]}\PY{p}{,}
        \PY{l+s+s1}{\PYZsq{}}\PY{l+s+s1}{V5}\PY{l+s+s1}{\PYZsq{}}\PY{p}{:} \PY{p}{[}\PY{l+m+mf}{4.45}\PY{p}{,}\PY{l+m+mf}{6.92}\PY{p}{,}\PY{l+m+mf}{4.28}\PY{p}{,}\PY{l+m+mf}{3.9}\PY{p}{,}\PY{l+m+mf}{5.5}\PY{p}{,}\PY{l+m+mf}{4.6}\PY{p}{,}\PY{l+m+mf}{5.62}\PY{p}{,}\PY{l+m+mf}{5.15}\PY{p}{,}\PY{l+m+mf}{6.18}\PY{p}{,}\PY{l+m+mf}{6.15}\PY{p}{,}\PY{l+m+mf}{5.88}\PY{p}{,}\PY{l+m+mf}{4.88}\PY{p}{,}\PY{l+m+mf}{5.5}\PY{p}{,}\PY{l+m+mi}{7}\PY{p}{,}\PY{l+m+mf}{10.78}\PY{p}{,}\PY{l+m+mf}{7.05}\PY{p}{,}\PY{l+m+mf}{6.35}\PY{p}{]}\PY{p}{,}
        \PY{l+s+s1}{\PYZsq{}}\PY{l+s+s1}{W}\PY{l+s+s1}{\PYZsq{}}\PY{p}{:} \PY{p}{[}\PY{l+m+mf}{556.52}\PY{p}{,}\PY{l+m+mf}{696.82}\PY{p}{,}\PY{l+m+mf}{1933.15}\PY{p}{,}\PY{l+m+mf}{1603.62}\PY{p}{,}\PY{l+m+mf}{1611.37}\PY{p}{,}\PY{l+m+mf}{1613.27}\PY{p}{,}\PY{l+m+mf}{1854.17}\PY{p}{,}\PY{l+m+mf}{2160.55}\PY{p}{,}\PY{l+m+mf}{2305.58}\PY{p}{,}\PY{l+m+mf}{3503.93}\PY{p}{,}\PY{l+m+mf}{3571.89}\PY{p}{,}\PY{l+m+mf}{3741.4}\PY{p}{,}\PY{l+m+mf}{4026.52}\PY{p}{,}\PY{l+m+mf}{10343.81}\PY{p}{,}\PY{l+m+mf}{11732.17}\PY{p}{,}\PY{l+m+mf}{15414.94}\PY{p}{,}\PY{l+m+mf}{18854.45}\PY{p}{]}
    \PY{p}{\PYZcb{}}
\PY{p}{)}
\PY{c+c1}{\PYZsh{} Exploración inicial de datos}
\PY{n}{display}\PY{p}{(}\PY{n}{df}\PY{p}{)}
\PY{n}{display}\PY{p}{(}\PY{n}{df}\PY{o}{.}\PY{n}{describe}\PY{p}{(}\PY{p}{)}\PY{p}{)}
\end{Verbatim}
\end{tcolorbox}

    
    \begin{Verbatim}[commandchars=\\\{\}]
        V1     V2        V3     V4     V5         W
0    15.57   2463    472.82   18.0   4.45    556.52
1    44.02   2048   1339.75    9.5   6.92    696.82
2    20.42   3940    620.25   12.8   4.28   1933.15
3    18.74   6505    568.33   36.7   3.90   1603.62
4    49.20   5723   1497.60   35.7   5.50   1611.37
5    44.92  11520   1365.83   24.0   4.60   1613.27
6    55.48   5779   1687.00   43.3   5.62   1854.17
7    59.28   5969   1639.92   46.7   5.15   2160.55
8    94.39   8461   2872.33   78.7   6.18   2305.58
9   128.02  20106   3655.08  180.5   6.15   3503.93
10   96.00  13313   2912.00   60.9   5.88   3571.89
11  131.42  10771   3921.00  103.7   4.88   3741.40
12  127.21  15543   3865.67  126.8   5.50   4026.52
13  252.90  36194   7684.10  157.7   7.00  10343.81
14  409.20  34703  12246.33  169.4  10.78  11732.17
15  463.70  39204  14098.40  331.4   7.05  15414.94
16  510.22  86533  15524.00  371.6   6.35  18854.45
    \end{Verbatim}

    
    
    \begin{Verbatim}[commandchars=\\\{\}]
               V1            V2            V3          V4         V5  \textbackslash{}
count   17.000000     17.000000     17.000000   17.000000  17.000000   
mean   148.275882  18163.235294   4468.847647  106.317647   5.893529   
std    161.038581  21278.110550   4886.552611  107.954153   1.584073   
min     15.570000   2048.000000    472.820000    9.500000   3.900000   
25\%     44.920000   5779.000000   1365.830000   35.700000   4.880000   
50\%     94.390000  10771.000000   2872.330000   60.900000   5.620000   
75\%    131.420000  20106.000000   3921.000000  157.700000   6.350000   
max    510.220000  86533.000000  15524.000000  371.600000  10.780000   

                  W  
count     17.000000  
mean    5030.832941  
std     5525.298021  
min      556.520000  
25\%     1613.270000  
50\%     2305.580000  
75\%     4026.520000  
max    18854.450000  
    \end{Verbatim}

    
    \hypertarget{exploraciuxf3n-de-datos}{%
\paragraph{Exploración de datos}\label{exploraciuxf3n-de-datos}}

Previo a analizar los datos, conviene notar algunas observaciones que
pueden resultar relevantes para el estudio de los datos. No existen
datos faltantes, por lo que no se requieren técnicas para lidiar con
ellos. Puede observarse que la magnitud de varias variables, por ejemplo
entre \(V_3\) y \(V_4\) son muy distintas, lo cual de antemano puede
sugerir problemas de condicionamiento cuando se haga un modelo lineal.
Adicionalmente, del análisis de estadísticas descriptivas, puede
observarse que el valor máximo es mucho mayor que el valor inicial del
cuarto cuartil, lo que puede sugerir que las distribuciones de los
puntos observados tengan colas muy largas. \#\#\#\# Análisis de
colinealidad

\hypertarget{inciso-1.}{%
\subparagraph{Inciso 1.}\label{inciso-1.}}

En los gráficos de dispersión entre variables puede observarse que, en
general, existe una fuerte correlación entre las variables predictoras,
siendo prácticamente perfecta entre \(V_1\) y \(V_3\). Observando que la
variable \(V_1\) representa la carga \emph{promedio} diaria de
pacientes, y \(V_3\) el número de días-cama de ocupación mensual, hace
sentido que tengan una relación muy fuerte, pues ambas dependen de la
cantidad de pacientes que hubieron. La gran diferencia entre ellos es
que un paciente puede salir en menos de un día. Esto quiere decir que si
se divide el número de días-cama entre la carga promedio de pacientes
diaria, se esperaría que tuvieran una relación cercana a 30, pero nunca
mayor a 31 (pues no hay meses de más de 31 días). Tampoco resulta de
sorprender que esté relacionado el número de pacientes con el número de
solicitudes de rayos X mensualmente (\(V_2\)), ni con la población
elegible.

Vale la pena notarse que las distribuciones de las variables,
efectivamente, no muestran un comportamiento normal, por lo que puede
valer la pena realizar una transformación{[}\^{}2{]} de los datos para
ver si después de transformarlos muestran este comportamiento normal.
Esto se discutirá al final de la tarea, tras haberse resuelto todos los
incisos.

El código utilizado se muestra a continuación. {[}\^{}2{]}: Se propone
una transformación Box-Cox, pues todos los valores son estrictamente
positivos.

    \begin{tcolorbox}[breakable, size=fbox, boxrule=1pt, pad at break*=1mm,colback=cellbackground, colframe=cellborder]
\prompt{In}{incolor}{5}{\boxspacing}
\begin{Verbatim}[commandchars=\\\{\}]
\PY{c+c1}{\PYZsh{} Análisis de colinealidad.}
\PY{c+c1}{\PYZsh{}\PYZsh{} Inciso 1.}
\PY{n}{sns}\PY{o}{.}\PY{n}{pairplot}\PY{p}{(}\PY{n}{df}\PY{p}{,}\PY{n}{corner}\PY{o}{=}\PY{k+kc}{True}\PY{p}{,}\PY{n}{kind}\PY{o}{=}\PY{l+s+s1}{\PYZsq{}}\PY{l+s+s1}{reg}\PY{l+s+s1}{\PYZsq{}}\PY{p}{)}\PY{o}{.}\PY{n}{fig}\PY{o}{.}\PY{n}{suptitle}\PY{p}{(}\PY{l+s+s1}{\PYZsq{}}\PY{l+s+s1}{Gráficos de dispersión para el conjunto de datos.}\PY{l+s+s1}{\PYZsq{}}\PY{p}{)}
\PY{n}{plt}\PY{o}{.}\PY{n}{figure}\PY{p}{(}\PY{p}{)}
\PY{n}{sns}\PY{o}{.}\PY{n}{heatmap}\PY{p}{(}\PY{n}{df}\PY{o}{.}\PY{n}{corr}\PY{p}{(}\PY{p}{)}\PY{p}{,}\PY{n}{cmap}\PY{o}{=}\PY{l+s+s1}{\PYZsq{}}\PY{l+s+s1}{Blues}\PY{l+s+s1}{\PYZsq{}}\PY{p}{)}\PY{o}{.}\PY{n}{set}\PY{p}{(}\PY{n}{title}\PY{o}{=}\PY{l+s+s1}{\PYZsq{}}\PY{l+s+s1}{Matriz de correlación de variables \PYZdl{}V\PYZus{}1\PYZdl{} a \PYZdl{}V\PYZus{}5\PYZdl{} y \PYZdl{}W\PYZdl{}}\PY{l+s+s1}{\PYZsq{}}\PY{p}{)}
\PY{n}{plt}\PY{o}{.}\PY{n}{figure}\PY{p}{(}\PY{p}{)}
\PY{n}{sns}\PY{o}{.}\PY{n}{kdeplot}\PY{p}{(}\PY{n}{df}\PY{o}{.}\PY{n}{V3}\PY{o}{/}\PY{n}{df}\PY{o}{.}\PY{n}{V1}\PY{p}{,}\PY{n}{bw\PYZus{}method}\PY{o}{=}\PY{l+m+mf}{0.3}\PY{p}{)}
\PY{n}{plt}\PY{o}{.}\PY{n}{title}\PY{p}{(}\PY{l+s+s1}{\PYZsq{}}\PY{l+s+s1}{Estimación de densidad para relación entre variables \PYZdl{}V\PYZus{}3\PYZdl{} y \PYZdl{}V\PYZus{}1\PYZdl{}}\PY{l+s+s1}{\PYZsq{}}\PY{p}{)}
\end{Verbatim}
\end{tcolorbox}

            \begin{tcolorbox}[breakable, size=fbox, boxrule=.5pt, pad at break*=1mm, opacityfill=0]
\prompt{Out}{outcolor}{5}{\boxspacing}
\begin{Verbatim}[commandchars=\\\{\}]
Text(0.5, 1.0, 'Estimación de densidad para relación entre variables \$V\_3\$ y
\$V\_1\$')
\end{Verbatim}
\end{tcolorbox}
        
    \begin{center}
    \adjustimage{max size={0.9\linewidth}{0.9\paperheight}}{Tarea1_JLAC_files/Tarea1_JLAC_20_1.png}
    \end{center}
    { \hspace*{\fill} \\}
    
    \begin{center}
    \adjustimage{max size={0.9\linewidth}{0.9\paperheight}}{Tarea1_JLAC_files/Tarea1_JLAC_20_2.png}
    \end{center}
    { \hspace*{\fill} \\}
    
    \begin{center}
    \adjustimage{max size={0.9\linewidth}{0.9\paperheight}}{Tarea1_JLAC_files/Tarea1_JLAC_20_3.png}
    \end{center}
    { \hspace*{\fill} \\}
    
    Se verifica que, efectivamente, la relación entre el número de pacientes
diarios promedio y los días-cama es de aproximadamente 30, por lo que
puede considerarse eliminar una de estas variables, ya que conllevan
prácticamente la misma información (número de pacientes atendidos).
Antes de eliminarlos, sin embargo, se realiza el análisis completo de
colinealidad, y posteriormente se decide qué hacer con estos valores.

\hypertarget{inciso-2.}{%
\subparagraph{Inciso 2.}\label{inciso-2.}}

Puede observarse que el número de condición de la matriz es muy elevado,
lo que elucida una diferencia muy grande entre el mayor y menor valor
singular de la matriz de datos, y sugiere problemas numéricos. También
se ve elevado por la diferencia en escalas de las distintas variables,
pero esto fue mencionado anteriormente.

    \begin{tcolorbox}[breakable, size=fbox, boxrule=1pt, pad at break*=1mm,colback=cellbackground, colframe=cellborder]
\prompt{In}{incolor}{6}{\boxspacing}
\begin{Verbatim}[commandchars=\\\{\}]
\PY{c+c1}{\PYZsh{}\PYZsh{} Inciso 2.}
\PY{n}{X} \PY{o}{=} \PY{n}{df}\PY{o}{.}\PY{n}{drop}\PY{p}{(}\PY{l+s+s1}{\PYZsq{}}\PY{l+s+s1}{W}\PY{l+s+s1}{\PYZsq{}}\PY{p}{,}\PY{n}{axis}\PY{o}{=}\PY{l+m+mi}{1}\PY{p}{)}\PY{o}{.}\PY{n}{to\PYZus{}numpy}\PY{p}{(}\PY{p}{)}
\PY{n}{Xcond} \PY{o}{=} \PY{n}{np}\PY{o}{.}\PY{n}{linalg}\PY{o}{.}\PY{n}{cond}\PY{p}{(}\PY{n}{X}\PY{p}{)}
\PY{n+nb}{print}\PY{p}{(}\PY{l+s+sa}{f}\PY{l+s+s2}{\PYZdq{}}\PY{l+s+s2}{Número de condición de la matriz X: }\PY{l+s+si}{\PYZob{}}\PY{n}{Xcond}\PY{l+s+si}{:}\PY{l+s+s2}{.3e}\PY{l+s+si}{\PYZcb{}}\PY{l+s+s2}{\PYZdq{}}\PY{p}{)}
\end{Verbatim}
\end{tcolorbox}

    \begin{Verbatim}[commandchars=\\\{\}]
Número de condición de la matriz X: 1.270e+04
    \end{Verbatim}

    \hypertarget{inciso-3.}{%
\subparagraph{Inciso 3.}\label{inciso-3.}}

Puede observarse que, cuando se escalan y centran los datos, el número
de condición se reduce en casi dos órdenes de magnitud, llegando a un
valor de cerca de 200, que si bien sigue sugiriendo la posible presencia
de colinealidad fuerte, el número de condición es mucho menor que antes.

    \begin{tcolorbox}[breakable, size=fbox, boxrule=1pt, pad at break*=1mm,colback=cellbackground, colframe=cellborder]
\prompt{In}{incolor}{7}{\boxspacing}
\begin{Verbatim}[commandchars=\\\{\}]
\PY{c+c1}{\PYZsh{}\PYZsh{} Inciso 3.}
\PY{n}{Xc} \PY{o}{=} \PY{n}{StandardScaler}\PY{p}{(}\PY{p}{)}\PY{o}{.}\PY{n}{fit\PYZus{}transform}\PY{p}{(}\PY{n}{X}\PY{p}{)}
\PY{n}{Xcond} \PY{o}{=} \PY{n}{np}\PY{o}{.}\PY{n}{linalg}\PY{o}{.}\PY{n}{cond}\PY{p}{(}\PY{n}{Xc}\PY{p}{)}
\PY{n+nb}{print}\PY{p}{(}\PY{l+s+sa}{f}\PY{l+s+s2}{\PYZdq{}}\PY{l+s+s2}{Número de condición de la matriz Xc: }\PY{l+s+si}{\PYZob{}}\PY{n}{Xcond}\PY{l+s+si}{:}\PY{l+s+s2}{.3f}\PY{l+s+si}{\PYZcb{}}\PY{l+s+s2}{\PYZdq{}}\PY{p}{)}
\end{Verbatim}
\end{tcolorbox}

    \begin{Verbatim}[commandchars=\\\{\}]
Número de condición de la matriz Xc: 222.643
    \end{Verbatim}

    \hypertarget{inciso-4.}{%
\subparagraph{Inciso 4.}\label{inciso-4.}}

Puede observarse de los factores de inflación de varianza que el mayor
es de 357.83, y el segundo mayor de 337.35, correspondientes a la
primera y tercera variable, por lo que se espera que en el modelo por
mínimos cuadrados la varianza de los estimadores de estas variables sean
muy elevadas, pudiendo llegar a provocar que se crea que no son
estadísticamente significativas por la presencia de colinealidad.

    \begin{tcolorbox}[breakable, size=fbox, boxrule=1pt, pad at break*=1mm,colback=cellbackground, colframe=cellborder]
\prompt{In}{incolor}{8}{\boxspacing}
\begin{Verbatim}[commandchars=\\\{\}]
\PY{c+c1}{\PYZsh{}\PYZsh{} Inciso 4.}
\PY{n}{R} \PY{o}{=} \PY{n}{Xc}\PY{o}{.}\PY{n}{T} \PY{o}{@} \PY{n}{Xc}
\PY{n}{VIF} \PY{o}{=} \PY{n}{np}\PY{o}{.}\PY{n}{diag}\PY{p}{(}\PY{n}{np}\PY{o}{.}\PY{n}{linalg}\PY{o}{.}\PY{n}{inv}\PY{p}{(}\PY{n}{R}\PY{p}{)}\PY{p}{)}
\PY{k}{with} \PY{n}{np}\PY{o}{.}\PY{n}{printoptions}\PY{p}{(}\PY{n}{precision}\PY{o}{=}\PY{l+m+mi}{2}\PY{p}{,}\PY{n}{suppress}\PY{o}{=}\PY{k+kc}{True}\PY{p}{)}\PY{p}{:}
    \PY{n+nb}{print}\PY{p}{(}\PY{l+s+sa}{f}\PY{l+s+s1}{\PYZsq{}}\PY{l+s+s1}{Factores de inflación de la varianza: }\PY{l+s+si}{\PYZob{}}\PY{n}{VIF}\PY{l+s+si}{\PYZcb{}}\PY{l+s+s1}{\PYZsq{}}\PY{p}{)}
\end{Verbatim}
\end{tcolorbox}

    \begin{Verbatim}[commandchars=\\\{\}]
Factores de inflación de la varianza: [357.83   0.47 337.35   0.96   0.28]
    \end{Verbatim}

    \hypertarget{inciso-5.}{%
\subparagraph{Inciso 5.}\label{inciso-5.}}

Puede observarse en la matriz \(\Pi\) que, efectivamente, existe una
fuerte colinealidad asociada al cuarto número de condición, que,
partiendo de los análisis anteriores, puede concluirse que es entre las
variables \(V_1\) y \(V_3\). La presencia de esta colinealidad puede
aumentar fuertemente la varianza de los estimadores del modelo lineal
por mínimos cuadrados, por lo que se sugiere o eliminar una de las
variables del modelo, o utilizar regresión regularizada (por ejemplo,
Lasso o Ridge).

    \begin{tcolorbox}[breakable, size=fbox, boxrule=1pt, pad at break*=1mm,colback=cellbackground, colframe=cellborder]
\prompt{In}{incolor}{9}{\boxspacing}
\begin{Verbatim}[commandchars=\\\{\}]
\PY{c+c1}{\PYZsh{}\PYZsh{} Inciso 5.}
\PY{n}{U}\PY{p}{,} \PY{n}{D}\PY{p}{,} \PY{n}{Vh} \PY{o}{=} \PY{n}{np}\PY{o}{.}\PY{n}{linalg}\PY{o}{.}\PY{n}{svd}\PY{p}{(}\PY{n}{Xc}\PY{p}{)} \PY{c+c1}{\PYZsh{} NOTA: Vh = V.T}
\PY{n}{phis} \PY{o}{=} \PY{n}{Vh}\PY{o}{.}\PY{n}{T}\PY{o}{*}\PY{o}{*}\PY{l+m+mi}{2}\PY{o}{/}\PY{n}{D}\PY{o}{*}\PY{o}{*}\PY{l+m+mi}{2}
\PY{n}{pis} \PY{o}{=} \PY{n}{phis}\PY{o}{.}\PY{n}{T}\PY{o}{/}\PY{n+nb}{sum}\PY{p}{(}\PY{n}{phis}\PY{o}{.}\PY{n}{T}\PY{p}{)}

\PY{n}{sns}\PY{o}{.}\PY{n}{heatmap}\PY{p}{(}\PY{n}{pis}\PY{p}{,}\PY{n}{cmap}\PY{o}{=}\PY{l+s+s1}{\PYZsq{}}\PY{l+s+s1}{Blues}\PY{l+s+s1}{\PYZsq{}}\PY{p}{)}
\PY{n}{plt}\PY{o}{.}\PY{n}{title}\PY{p}{(}\PY{l+s+s1}{\PYZsq{}}\PY{l+s+s1}{Proporciones de la descomposición de las varianzas.}\PY{l+s+s1}{\PYZsq{}}\PY{p}{)}
\PY{n}{plt}\PY{o}{.}\PY{n}{ylabel}\PY{p}{(}\PY{l+s+s1}{\PYZsq{}}\PY{l+s+s1}{Número de condición}\PY{l+s+s1}{\PYZsq{}}\PY{p}{)}
\PY{n}{plt}\PY{o}{.}\PY{n}{xlabel}\PY{p}{(}\PY{l+s+s1}{\PYZsq{}}\PY{l+s+s1}{Estimador}\PY{l+s+s1}{\PYZsq{}}\PY{p}{)}
\end{Verbatim}
\end{tcolorbox}

            \begin{tcolorbox}[breakable, size=fbox, boxrule=.5pt, pad at break*=1mm, opacityfill=0]
\prompt{Out}{outcolor}{9}{\boxspacing}
\begin{Verbatim}[commandchars=\\\{\}]
Text(0.5, 12.5, 'Estimador')
\end{Verbatim}
\end{tcolorbox}
        
    \begin{center}
    \adjustimage{max size={0.9\linewidth}{0.9\paperheight}}{Tarea1_JLAC_files/Tarea1_JLAC_28_1.png}
    \end{center}
    { \hspace*{\fill} \\}
    
    \hypertarget{regresiuxf3n-ridge}{%
\paragraph{Regresión ridge}\label{regresiuxf3n-ridge}}

\hypertarget{inciso-1.}{%
\subparagraph{Inciso 1.}\label{inciso-1.}}

Utilizando el estimador del parámetro de Ridge propuesto, se obtiene el
siguiente valor:

    \begin{tcolorbox}[breakable, size=fbox, boxrule=1pt, pad at break*=1mm,colback=cellbackground, colframe=cellborder]
\prompt{In}{incolor}{10}{\boxspacing}
\begin{Verbatim}[commandchars=\\\{\}]
\PY{n}{p} \PY{o}{=} \PY{l+m+mi}{5} \PY{c+c1}{\PYZsh{} Número de parámetros a utilizar}
\PY{n}{n} \PY{o}{=} \PY{n+nb}{len}\PY{p}{(}\PY{n}{Xc}\PY{p}{)}
\PY{n}{res} \PY{o}{=} \PY{n}{np}\PY{o}{.}\PY{n}{linalg}\PY{o}{.}\PY{n}{lstsq}\PY{p}{(}\PY{n}{Xc}\PY{p}{,}\PY{n}{df}\PY{p}{[}\PY{l+s+s1}{\PYZsq{}}\PY{l+s+s1}{W}\PY{l+s+s1}{\PYZsq{}}\PY{p}{]}\PY{o}{.}\PY{n}{to\PYZus{}numpy}\PY{p}{(}\PY{p}{)}\PY{o}{.}\PY{n}{T}\PY{p}{,}\PY{n}{rcond}\PY{o}{=}\PY{o}{\PYZhy{}}\PY{l+m+mi}{1}\PY{p}{)}
\PY{n}{beta} \PY{o}{=} \PY{n}{res}\PY{p}{[}\PY{l+m+mi}{0}\PY{p}{]}
\PY{n}{SSR} \PY{o}{=} \PY{n}{res}\PY{p}{[}\PY{l+m+mi}{1}\PY{p}{]}
\PY{n}{sigma2} \PY{o}{=} \PY{n}{SSR}\PY{o}{/}\PY{p}{(}\PY{n}{n}\PY{o}{\PYZhy{}}\PY{n}{p}\PY{p}{)}
\PY{n}{lamb} \PY{o}{=} \PY{n}{p}\PY{o}{*}\PY{n}{sigma2}\PY{o}{/}\PY{n+nb}{sum}\PY{p}{(}\PY{n}{beta}\PY{o}{*}\PY{o}{*}\PY{l+m+mi}{2}\PY{p}{)}
\PY{n+nb}{print}\PY{p}{(}\PY{l+s+sa}{f}\PY{l+s+s1}{\PYZsq{}}\PY{l+s+s1}{Parámetro de Ridge estimado: }\PY{l+s+si}{\PYZob{}}\PY{n}{lamb}\PY{p}{[}\PY{l+m+mi}{0}\PY{p}{]}\PY{l+s+si}{:}\PY{l+s+s1}{.4f}\PY{l+s+si}{\PYZcb{}}\PY{l+s+s1}{\PYZsq{}}\PY{p}{)}
\end{Verbatim}
\end{tcolorbox}

    \begin{Verbatim}[commandchars=\\\{\}]
Parámetro de Ridge estimado: 1.0156
    \end{Verbatim}

    \hypertarget{inciso-2.}{%
\subparagraph{Inciso 2.}\label{inciso-2.}}

Para el método de validación cruzada, se realiza con ayuda de los
modelos Ridge de sklearn. Puede observarse que el resultado de la
validación cruzada cuando solamente se deja un valor fuera contra si se
hace un 5-fold CV muestra dos valores de parámetro de Ridge sumamente
distintos. En la gráfica en la que se muestra el error cuadrático medio
contra \(\lambda\) para cada valor removido, puede observarse que existe
un punto sumamente influyente, por lo que hace que se desplace tanto en
el parámetro de Ridge estimado con el estimador de mínimos cuadrados
usando todos los valores, como por validación cruzada. Esto sugiere que
existe un punto sumamente influyente en la regresión, que tenga mucho
``apalancamiento'' sobre el valor de los estimadores; sospecha que se
observa desde el primer inciso en el análisis de colinealidad -- incluso
desde el análisis previo -- al notar que la distribución tiene valores
sumamente alejados de las medidas de tendencia central. Por esto, se
toma el valor de parámetro de Ridge menor, y se considera la opción de
transformar los datos y no solo estandarizarlos. Este ejercicio se verá
tras ser respondidos todos los incisos.

    \begin{tcolorbox}[breakable, size=fbox, boxrule=1pt, pad at break*=1mm,colback=cellbackground, colframe=cellborder]
\prompt{In}{incolor}{11}{\boxspacing}
\begin{Verbatim}[commandchars=\\\{\}]
\PY{n}{lambdas} \PY{o}{=} \PY{n}{np}\PY{o}{.}\PY{n}{logspace}\PY{p}{(}\PY{o}{\PYZhy{}}\PY{l+m+mi}{5}\PY{p}{,}\PY{l+m+mi}{1}\PY{p}{)}
\PY{c+c1}{\PYZsh{} Modelo con LOOCV}
\PY{n}{ridge} \PY{o}{=} \PY{n}{RidgeCV}\PY{p}{(}\PY{n}{lambdas}\PY{p}{,}\PY{n}{store\PYZus{}cv\PYZus{}values}\PY{o}{=}\PY{k+kc}{True}\PY{p}{)}\PY{o}{.}\PY{n}{fit}\PY{p}{(}\PY{n}{Xc}\PY{p}{,}\PY{n}{df}\PY{p}{[}\PY{l+s+s1}{\PYZsq{}}\PY{l+s+s1}{W}\PY{l+s+s1}{\PYZsq{}}\PY{p}{]}\PY{p}{)}
\PY{n}{plt}\PY{o}{.}\PY{n}{figure}\PY{p}{(}\PY{n}{figsize}\PY{o}{=}\PY{p}{(}\PY{l+m+mi}{16}\PY{p}{,}\PY{l+m+mi}{9}\PY{p}{)}\PY{p}{)}
\PY{n}{plt}\PY{o}{.}\PY{n}{plot}\PY{p}{(}\PY{n}{lambdas}\PY{p}{,}\PY{n}{ridge}\PY{o}{.}\PY{n}{cv\PYZus{}values\PYZus{}}\PY{o}{.}\PY{n}{T}\PY{p}{)}
\PY{n}{plt}\PY{o}{.}\PY{n}{xscale}\PY{p}{(}\PY{l+s+s1}{\PYZsq{}}\PY{l+s+s1}{log}\PY{l+s+s1}{\PYZsq{}}\PY{p}{)}
\PY{n}{plt}\PY{o}{.}\PY{n}{title}\PY{p}{(}\PY{l+s+s1}{\PYZsq{}}\PY{l+s+s1}{Errores cuadráticos medios de cada validación cruzada con cada valor removido}\PY{l+s+s1}{\PYZsq{}}\PY{p}{)}
\PY{n}{plt}\PY{o}{.}\PY{n}{xlabel}\PY{p}{(}\PY{l+s+s1}{\PYZsq{}}\PY{l+s+s1}{\PYZdl{}}\PY{l+s+s1}{\PYZbs{}}\PY{l+s+s1}{lambda\PYZdl{}}\PY{l+s+s1}{\PYZsq{}}\PY{p}{)}
\PY{n}{plt}\PY{o}{.}\PY{n}{ylabel}\PY{p}{(}\PY{l+s+s1}{\PYZsq{}}\PY{l+s+s1}{MSE (horas de trabajo/mes)}\PY{l+s+s1}{\PYZsq{}}\PY{p}{)}
\PY{n}{plt}\PY{o}{.}\PY{n}{legend}\PY{p}{(}\PY{n}{df}\PY{o}{.}\PY{n}{index}\PY{o}{.}\PY{n}{to\PYZus{}numpy}\PY{p}{(}\PY{p}{)}\PY{p}{,}\PY{n}{loc}\PY{o}{=}\PY{l+s+s1}{\PYZsq{}}\PY{l+s+s1}{upper left}\PY{l+s+s1}{\PYZsq{}}\PY{p}{,}\PY{n}{ncol}\PY{o}{=}\PY{l+m+mi}{6}\PY{p}{)}
\PY{n+nb}{print}\PY{p}{(}\PY{l+s+s1}{\PYZsq{}}\PY{l+s+s1}{Con validación cruzada Leave One Out}\PY{l+s+s1}{\PYZsq{}}\PY{p}{)}
\PY{n+nb}{print}\PY{p}{(}\PY{l+s+sa}{f}\PY{l+s+s1}{\PYZsq{}}\PY{l+s+s1}{Error cuadrático medio: }\PY{l+s+si}{\PYZob{}}\PY{n}{mean\PYZus{}squared\PYZus{}error}\PY{p}{(}\PY{n}{df}\PY{p}{[}\PY{l+s+s2}{\PYZdq{}}\PY{l+s+s2}{W}\PY{l+s+s2}{\PYZdq{}}\PY{p}{]}\PY{p}{,}\PY{n}{ridge}\PY{o}{.}\PY{n}{predict}\PY{p}{(}\PY{n}{Xc}\PY{p}{)}\PY{p}{)}\PY{l+s+si}{\PYZcb{}}\PY{l+s+s1}{\PYZsq{}}\PY{p}{)}
\PY{n+nb}{print}\PY{p}{(}\PY{l+s+sa}{f}\PY{l+s+s1}{\PYZsq{}}\PY{l+s+s1}{Valor del parámetro de Ridge: }\PY{l+s+si}{\PYZob{}}\PY{n}{ridge}\PY{o}{.}\PY{n}{alpha\PYZus{}}\PY{l+s+si}{\PYZcb{}}\PY{l+s+s1}{\PYZsq{}}\PY{p}{)}

\PY{c+c1}{\PYZsh{} Modelo con 5\PYZhy{}Fold CV}
\PY{n}{folder} \PY{o}{=} \PY{n}{KFold}\PY{p}{(}\PY{l+m+mi}{5}\PY{p}{,}\PY{n}{shuffle}\PY{o}{=}\PY{k+kc}{True}\PY{p}{,}\PY{n}{random\PYZus{}state}\PY{o}{=}\PY{l+m+mi}{9}\PY{p}{)}
\PY{n}{ridge\PYZus{}2} \PY{o}{=} \PY{n}{RidgeCV}\PY{p}{(}\PY{n}{lambdas}\PY{p}{,}\PY{n}{cv}\PY{o}{=}\PY{n}{folder}\PY{p}{)}\PY{o}{.}\PY{n}{fit}\PY{p}{(}\PY{n}{Xc}\PY{p}{,}\PY{n}{df}\PY{p}{[}\PY{l+s+s1}{\PYZsq{}}\PY{l+s+s1}{W}\PY{l+s+s1}{\PYZsq{}}\PY{p}{]}\PY{p}{)}
\PY{n+nb}{print}\PY{p}{(}\PY{l+s+s1}{\PYZsq{}}\PY{l+s+s1}{Con 5\PYZhy{}fold CV:}\PY{l+s+s1}{\PYZsq{}}\PY{p}{)}
\PY{n+nb}{print}\PY{p}{(}\PY{l+s+sa}{f}\PY{l+s+s1}{\PYZsq{}}\PY{l+s+s1}{Error cuadrático medio: }\PY{l+s+si}{\PYZob{}}\PY{n}{mean\PYZus{}squared\PYZus{}error}\PY{p}{(}\PY{n}{df}\PY{p}{[}\PY{l+s+s2}{\PYZdq{}}\PY{l+s+s2}{W}\PY{l+s+s2}{\PYZdq{}}\PY{p}{]}\PY{p}{,}\PY{n}{ridge\PYZus{}2}\PY{o}{.}\PY{n}{predict}\PY{p}{(}\PY{n}{Xc}\PY{p}{)}\PY{p}{)}\PY{l+s+si}{\PYZcb{}}\PY{l+s+s1}{\PYZsq{}}\PY{p}{)}
\PY{n+nb}{print}\PY{p}{(}\PY{l+s+sa}{f}\PY{l+s+s1}{\PYZsq{}}\PY{l+s+s1}{Valor del parámetro de Ridge: }\PY{l+s+si}{\PYZob{}}\PY{n}{ridge\PYZus{}2}\PY{o}{.}\PY{n}{alpha\PYZus{}}\PY{l+s+si}{\PYZcb{}}\PY{l+s+s1}{\PYZsq{}}\PY{p}{)}
\end{Verbatim}
\end{tcolorbox}

    \begin{Verbatim}[commandchars=\\\{\}]
Con validación cruzada Leave One Out
Error cuadrático medio: 500846.57397637964
Valor del parámetro de Ridge: 1.389495494373136
Con 5-fold CV:
Error cuadrático medio: 325283.5424489137
Valor del parámetro de Ridge: 0.2559547922699533
    \end{Verbatim}

    \begin{center}
    \adjustimage{max size={0.9\linewidth}{0.9\paperheight}}{Tarea1_JLAC_files/Tarea1_JLAC_32_1.png}
    \end{center}
    { \hspace*{\fill} \\}
    
    \hypertarget{inciso-3.}{%
\subparagraph{Inciso 3.}\label{inciso-3.}}

A continuación se muestran los valores predichos por los modelos, así
como sus parámetros en la traza de Ridge. Se añade adicionalmente una
estimación de la distribución de los residuales con cada regresión.

    \begin{tcolorbox}[breakable, size=fbox, boxrule=1pt, pad at break*=1mm,colback=cellbackground, colframe=cellborder]
\prompt{In}{incolor}{12}{\boxspacing}
\begin{Verbatim}[commandchars=\\\{\}]
\PY{n}{coefs} \PY{o}{=} \PY{p}{[}\PY{p}{]}
\PY{n}{scores} \PY{o}{=} \PY{p}{[}\PY{p}{]}
\PY{k}{for} \PY{n}{value} \PY{o+ow}{in} \PY{n}{lambdas}\PY{p}{:}
    \PY{n}{model} \PY{o}{=} \PY{n}{Ridge}\PY{p}{(}\PY{n}{value}\PY{p}{)}\PY{o}{.}\PY{n}{fit}\PY{p}{(}\PY{n}{Xc}\PY{p}{,}\PY{n}{df}\PY{p}{[}\PY{l+s+s1}{\PYZsq{}}\PY{l+s+s1}{W}\PY{l+s+s1}{\PYZsq{}}\PY{p}{]}\PY{p}{)}
    \PY{n}{coefs}\PY{o}{.}\PY{n}{append}\PY{p}{(}\PY{n}{model}\PY{o}{.}\PY{n}{coef\PYZus{}}\PY{p}{)}
\PY{n}{plt}\PY{o}{.}\PY{n}{figure}\PY{p}{(}\PY{n}{figsize}\PY{o}{=}\PY{p}{(}\PY{l+m+mi}{8}\PY{p}{,}\PY{l+m+mi}{6}\PY{p}{)}\PY{p}{)}
\PY{n}{plt}\PY{o}{.}\PY{n}{plot}\PY{p}{(}\PY{n}{lambdas}\PY{p}{,}\PY{n}{coefs}\PY{p}{)}
\PY{n}{plt}\PY{o}{.}\PY{n}{xscale}\PY{p}{(}\PY{l+s+s1}{\PYZsq{}}\PY{l+s+s1}{log}\PY{l+s+s1}{\PYZsq{}}\PY{p}{)}
\PY{n}{plt}\PY{o}{.}\PY{n}{xlabel}\PY{p}{(}\PY{l+s+s1}{\PYZsq{}}\PY{l+s+s1}{\PYZdl{}}\PY{l+s+s1}{\PYZbs{}}\PY{l+s+s1}{lambda\PYZdl{}}\PY{l+s+s1}{\PYZsq{}}\PY{p}{)}
\PY{n}{plt}\PY{o}{.}\PY{n}{ylabel}\PY{p}{(}\PY{l+s+s1}{\PYZsq{}}\PY{l+s+s1}{\PYZdl{}}\PY{l+s+se}{\PYZbs{}\PYZbs{}}\PY{l+s+s1}{beta\PYZus{}i\PYZdl{}}\PY{l+s+s1}{\PYZsq{}}\PY{p}{)}
\PY{n}{plt}\PY{o}{.}\PY{n}{legend}\PY{p}{(}\PY{p}{[}\PY{l+s+s1}{\PYZsq{}}\PY{l+s+s1}{\PYZdl{}}\PY{l+s+se}{\PYZbs{}\PYZbs{}}\PY{l+s+s1}{beta\PYZus{}}\PY{l+s+si}{\PYZob{}V\PYZus{}1\PYZcb{}}\PY{l+s+s1}{\PYZdl{}}\PY{l+s+s1}{\PYZsq{}}\PY{p}{,}\PY{l+s+s1}{\PYZsq{}}\PY{l+s+s1}{\PYZdl{}}\PY{l+s+se}{\PYZbs{}\PYZbs{}}\PY{l+s+s1}{beta\PYZus{}}\PY{l+s+si}{\PYZob{}V\PYZus{}2\PYZcb{}}\PY{l+s+s1}{\PYZdl{}}\PY{l+s+s1}{\PYZsq{}}\PY{p}{,}\PY{l+s+s1}{\PYZsq{}}\PY{l+s+s1}{\PYZdl{}}\PY{l+s+se}{\PYZbs{}\PYZbs{}}\PY{l+s+s1}{beta\PYZus{}}\PY{l+s+si}{\PYZob{}V\PYZus{}3\PYZcb{}}\PY{l+s+s1}{\PYZdl{}}\PY{l+s+s1}{\PYZsq{}}\PY{p}{,}\PY{l+s+s1}{\PYZsq{}}\PY{l+s+s1}{\PYZdl{}}\PY{l+s+se}{\PYZbs{}\PYZbs{}}\PY{l+s+s1}{beta\PYZus{}}\PY{l+s+si}{\PYZob{}V\PYZus{}4\PYZcb{}}\PY{l+s+s1}{\PYZdl{}}\PY{l+s+s1}{\PYZsq{}}\PY{p}{,}\PY{l+s+s1}{\PYZsq{}}\PY{l+s+s1}{\PYZdl{}}\PY{l+s+se}{\PYZbs{}\PYZbs{}}\PY{l+s+s1}{beta\PYZus{}}\PY{l+s+si}{\PYZob{}V\PYZus{}5\PYZcb{}}\PY{l+s+s1}{\PYZdl{}}\PY{l+s+s1}{\PYZsq{}}\PY{p}{]}\PY{p}{)}
\PY{n}{plt}\PY{o}{.}\PY{n}{ylim}\PY{p}{(}\PY{o}{\PYZhy{}}\PY{l+m+mi}{7000}\PY{p}{,}\PY{l+m+mi}{12500}\PY{p}{)}
\PY{n}{plt}\PY{o}{.}\PY{n}{grid}\PY{p}{(}\PY{n}{which}\PY{o}{=}\PY{l+s+s1}{\PYZsq{}}\PY{l+s+s1}{both}\PY{l+s+s1}{\PYZsq{}}\PY{p}{,}\PY{n}{alpha}\PY{o}{=}\PY{l+m+mf}{0.2}\PY{p}{)}
\PY{n}{plt}\PY{o}{.}\PY{n}{vlines}\PY{p}{(}
    \PY{n}{x} \PY{o}{=} \PY{p}{[}\PY{n}{lamb}\PY{p}{[}\PY{l+m+mi}{0}\PY{p}{]}\PY{p}{,} \PY{n}{ridge\PYZus{}2}\PY{o}{.}\PY{n}{alpha\PYZus{}}\PY{p}{,} \PY{n}{ridge}\PY{o}{.}\PY{n}{alpha\PYZus{}}\PY{p}{]}\PY{p}{,}
    \PY{n}{ymin} \PY{o}{=} \PY{o}{\PYZhy{}}\PY{l+m+mi}{7000}\PY{p}{,}
    \PY{n}{ymax} \PY{o}{=} \PY{l+m+mi}{12500}\PY{p}{,} 
    \PY{n}{label} \PY{o}{=} \PY{p}{[}\PY{l+s+s1}{\PYZsq{}}\PY{l+s+s1}{Estimado con ecuación}\PY{l+s+s1}{\PYZsq{}}\PY{p}{,}\PY{l+s+s1}{\PYZsq{}}\PY{l+s+s1}{Estimado con CV 5\PYZhy{}fold}\PY{l+s+s1}{\PYZsq{}}\PY{p}{,} \PY{l+s+s1}{\PYZsq{}}\PY{l+s+s1}{Estimado con LOOCV}\PY{l+s+s1}{\PYZsq{}}\PY{p}{]}\PY{p}{,}
    \PY{n}{colors} \PY{o}{=} \PY{l+s+s1}{\PYZsq{}}\PY{l+s+s1}{black}\PY{l+s+s1}{\PYZsq{}}\PY{p}{,}
    \PY{n}{alpha} \PY{o}{=} \PY{l+m+mf}{0.2}\PY{p}{,}
    \PY{n}{linestyles} \PY{o}{=} \PY{l+s+s1}{\PYZsq{}}\PY{l+s+s1}{dashed}\PY{l+s+s1}{\PYZsq{}}
    \PY{p}{)}
\PY{n}{plt}\PY{o}{.}\PY{n}{title}\PY{p}{(}\PY{l+s+s1}{\PYZsq{}}\PY{l+s+s1}{Traza de Ridge para regresión de datos del hospital.}\PY{l+s+s1}{\PYZsq{}}\PY{p}{)}

\PY{n}{Y} \PY{o}{=} \PY{n}{df}\PY{p}{[}\PY{l+s+s1}{\PYZsq{}}\PY{l+s+s1}{W}\PY{l+s+s1}{\PYZsq{}}\PY{p}{]}
\PY{n}{heur} \PY{o}{=} \PY{n}{Ridge}\PY{p}{(}\PY{n}{lamb}\PY{p}{[}\PY{l+m+mi}{0}\PY{p}{]}\PY{p}{)}\PY{o}{.}\PY{n}{fit}\PY{p}{(}\PY{n}{Xc}\PY{p}{,}\PY{n}{Y}\PY{p}{)}
\PY{n}{Y\PYZus{}ridge\PYZus{}h} \PY{o}{=} \PY{n}{heur}\PY{o}{.}\PY{n}{predict}\PY{p}{(}\PY{n}{Xc}\PY{p}{)}
\PY{n}{Y\PYZus{}ridge\PYZus{}1} \PY{o}{=} \PY{n}{ridge}\PY{o}{.}\PY{n}{predict}\PY{p}{(}\PY{n}{Xc}\PY{p}{)}
\PY{n}{Y\PYZus{}ridge\PYZus{}2} \PY{o}{=} \PY{n}{ridge\PYZus{}2}\PY{o}{.}\PY{n}{predict}\PY{p}{(}\PY{n}{Xc}\PY{p}{)}
\PY{n}{preds} \PY{o}{=} \PY{n}{pd}\PY{o}{.}\PY{n}{DataFrame}\PY{p}{(}\PY{n}{data}\PY{o}{=} \PY{p}{\PYZob{}}
    \PY{l+s+s1}{\PYZsq{}}\PY{l+s+s1}{W ridge heurística}\PY{l+s+s1}{\PYZsq{}}\PY{p}{:} \PY{n}{Y\PYZus{}ridge\PYZus{}h}\PY{p}{,}
    \PY{l+s+s1}{\PYZsq{}}\PY{l+s+s1}{W ridge LOOCV}\PY{l+s+s1}{\PYZsq{}}\PY{p}{:} \PY{n}{Y\PYZus{}ridge\PYZus{}1}\PY{p}{,}
    \PY{l+s+s1}{\PYZsq{}}\PY{l+s+s1}{W ridge 5\PYZhy{}fold}\PY{l+s+s1}{\PYZsq{}}\PY{p}{:} \PY{n}{Y\PYZus{}ridge\PYZus{}2}
\PY{p}{\PYZcb{}}\PY{p}{)}
\PY{n+nb}{print}\PY{p}{(}\PY{l+s+sa}{f}\PY{l+s+s1}{\PYZsq{}}\PY{l+s+s1}{Parámetros:}\PY{l+s+se}{\PYZbs{}n}\PY{l+s+s1}{ Ridge heurística: }\PY{l+s+si}{\PYZob{}}\PY{n}{heur}\PY{o}{.}\PY{n}{coef\PYZus{}}\PY{l+s+si}{\PYZcb{}}\PY{l+s+se}{\PYZbs{}n}\PY{l+s+s1}{ Ridge LOOCV: }\PY{l+s+si}{\PYZob{}}\PY{n}{ridge}\PY{o}{.}\PY{n}{coef\PYZus{}}\PY{l+s+si}{\PYZcb{}}\PY{l+s+se}{\PYZbs{}n}\PY{l+s+s1}{ Ridge 5\PYZhy{}fold\PYZus{} }\PY{l+s+si}{\PYZob{}}\PY{n}{ridge\PYZus{}2}\PY{o}{.}\PY{n}{coef\PYZus{}}\PY{l+s+si}{\PYZcb{}}\PY{l+s+s1}{\PYZsq{}}\PY{p}{)}
\PY{n}{display}\PY{p}{(}\PY{n}{preds}\PY{p}{)}
\PY{n}{preds}\PY{o}{.}\PY{n}{transform}\PY{p}{(}\PY{k}{lambda} \PY{n}{x}\PY{p}{:} \PY{n}{x}\PY{o}{\PYZhy{}}\PY{n}{Y}\PY{p}{)}\PY{o}{.}\PY{n}{plot}\PY{o}{.}\PY{n}{kde}\PY{p}{(}\PY{p}{)}
\PY{n}{plt}\PY{o}{.}\PY{n}{title}\PY{p}{(}\PY{l+s+s1}{\PYZsq{}}\PY{l+s+s1}{Estimación de densidad de residuales de regresiones}\PY{l+s+s1}{\PYZsq{}}\PY{p}{)}
\PY{n}{plt}\PY{o}{.}\PY{n}{xlabel}\PY{p}{(}\PY{l+s+s1}{\PYZsq{}}\PY{l+s+s1}{Residual (horas de trabajo/mes)}\PY{l+s+s1}{\PYZsq{}}\PY{p}{)}
\PY{n+nb}{print}\PY{p}{(}\PY{l+s+sa}{f}\PY{l+s+s1}{\PYZsq{}}\PY{l+s+s1}{Kurtosis en exceso de cada distribución de residuales:}\PY{l+s+se}{\PYZbs{}n}\PY{l+s+si}{\PYZob{}}\PY{n}{preds}\PY{o}{.}\PY{n}{transform}\PY{p}{(}\PY{k}{lambda} \PY{n}{x}\PY{p}{:} \PY{n}{x}\PY{o}{\PYZhy{}}\PY{n}{Y}\PY{p}{)}\PY{o}{.}\PY{n}{kurtosis}\PY{p}{(}\PY{p}{)}\PY{l+s+si}{\PYZcb{}}\PY{l+s+s1}{\PYZsq{}}\PY{p}{)}
\end{Verbatim}
\end{tcolorbox}

    \begin{Verbatim}[commandchars=\\\{\}]
Parámetros:
 Ridge heurística: [1789.51214424 1339.21595168 1864.64718793  439.50409058
-168.34325404]
 Ridge LOOCV: [1680.68197642 1346.36298023 1740.8702524   591.94206841
-89.36644251]
 Ridge 5-fold\_ [2244.86274015 1248.1296321  2430.60583907 -253.08903909
-492.27723077]
    \end{Verbatim}

    
    \begin{Verbatim}[commandchars=\\\{\}]
    W ridge heurística  W ridge LOOCV  W ridge 5-fold
0           707.954120     696.563467      801.712875
1          1041.654980    1102.231476      859.228087
2           914.118285     879.705927     1103.318373
3          1182.782343    1167.042227     1271.616396
4          1666.993584    1686.271751     1628.349638
5          1991.713891    1956.139355     2166.367039
6          1835.804976    1863.011607     1762.275126
7          1938.893107    1945.542579     1946.565694
8          3008.928737    3059.315125     2826.337592
9          4887.985776    5045.170147     4178.590185
10         3315.915604    3324.498943     3302.292592
11         4242.738453    4210.328481     4391.774542
12         4511.362179    4550.466801     4337.015273
13         8758.068871    8739.132041     8794.270820
14        11881.143192   11844.981484    12049.984083
15        14614.316187   14537.496195    14858.171785
16        19023.785714   18916.262392    19246.289902
    \end{Verbatim}

    
    \begin{Verbatim}[commandchars=\\\{\}]
Kurtosis en exceso de cada distribución de residuales:
W ridge heurística    1.003455
W ridge LOOCV         1.047082
W ridge 5-fold        1.646048
dtype: float64
    \end{Verbatim}

    \begin{center}
    \adjustimage{max size={0.9\linewidth}{0.9\paperheight}}{Tarea1_JLAC_files/Tarea1_JLAC_34_3.png}
    \end{center}
    { \hspace*{\fill} \\}
    
    \begin{center}
    \adjustimage{max size={0.9\linewidth}{0.9\paperheight}}{Tarea1_JLAC_files/Tarea1_JLAC_34_4.png}
    \end{center}
    { \hspace*{\fill} \\}
    
    Puede verse en el gráfico de estimación de densidad que aunque se
aproximan a una normal los residuales estimados, la kurtosis en exceso
es un poco mayor que una normal, lo que sugiere colas un poco más
pesadas.

\hypertarget{inciso-4.}{%
\subparagraph{Inciso 4.}\label{inciso-4.}}

A continuación se muestran los gráficos de \(Y_i\) contra
\(\widehat{Y}_i\) para cada una de las regresiones.

    \begin{tcolorbox}[breakable, size=fbox, boxrule=1pt, pad at break*=1mm,colback=cellbackground, colframe=cellborder]
\prompt{In}{incolor}{13}{\boxspacing}
\begin{Verbatim}[commandchars=\\\{\}]
\PY{n}{fig}\PY{p}{,} \PY{n}{axs} \PY{o}{=} \PY{n}{plt}\PY{o}{.}\PY{n}{subplots}\PY{p}{(}\PY{l+m+mi}{1}\PY{p}{,}\PY{l+m+mi}{3}\PY{p}{,}\PY{n}{sharex}\PY{o}{=}\PY{k+kc}{True}\PY{p}{,}\PY{n}{sharey}\PY{o}{=}\PY{k+kc}{True}\PY{p}{)}
\PY{n}{fig}\PY{o}{.}\PY{n}{set}\PY{p}{(}\PY{n}{figwidth}\PY{o}{=}\PY{l+m+mi}{15}\PY{p}{)}
\PY{n}{fig}\PY{o}{.}\PY{n}{suptitle}\PY{p}{(}\PY{l+s+s1}{\PYZsq{}}\PY{l+s+s1}{Gráficos de predicción contra valor real para cada regresión}\PY{l+s+s1}{\PYZsq{}}\PY{p}{)}
\PY{n}{lims} \PY{o}{=} \PY{p}{[}\PY{l+m+mi}{1}\PY{p}{,}\PY{l+m+mi}{20000}\PY{p}{]}
\PY{n}{axs}\PY{p}{[}\PY{l+m+mi}{0}\PY{p}{]}\PY{o}{.}\PY{n}{scatter}\PY{p}{(}\PY{n}{Y}\PY{p}{,}\PY{n}{Y\PYZus{}ridge\PYZus{}h}\PY{p}{)}
\PY{n}{axs}\PY{p}{[}\PY{l+m+mi}{0}\PY{p}{]}\PY{o}{.}\PY{n}{plot}\PY{p}{(}\PY{n}{lims}\PY{p}{,}\PY{n}{lims}\PY{p}{,}\PY{n}{alpha}\PY{o}{=}\PY{l+m+mf}{0.2}\PY{p}{,}\PY{n}{color}\PY{o}{=}\PY{l+s+s1}{\PYZsq{}}\PY{l+s+s1}{black}\PY{l+s+s1}{\PYZsq{}}\PY{p}{,}\PY{n}{linestyle}\PY{o}{=}\PY{l+s+s1}{\PYZsq{}}\PY{l+s+s1}{dashed}\PY{l+s+s1}{\PYZsq{}}\PY{p}{)}
\PY{n}{axs}\PY{p}{[}\PY{l+m+mi}{1}\PY{p}{]}\PY{o}{.}\PY{n}{scatter}\PY{p}{(}\PY{n}{Y}\PY{p}{,}\PY{n}{Y\PYZus{}ridge\PYZus{}1}\PY{p}{)}
\PY{n}{axs}\PY{p}{[}\PY{l+m+mi}{1}\PY{p}{]}\PY{o}{.}\PY{n}{plot}\PY{p}{(}\PY{n}{lims}\PY{p}{,}\PY{n}{lims}\PY{p}{,}\PY{n}{alpha}\PY{o}{=}\PY{l+m+mf}{0.2}\PY{p}{,}\PY{n}{color}\PY{o}{=}\PY{l+s+s1}{\PYZsq{}}\PY{l+s+s1}{black}\PY{l+s+s1}{\PYZsq{}}\PY{p}{,}\PY{n}{linestyle}\PY{o}{=}\PY{l+s+s1}{\PYZsq{}}\PY{l+s+s1}{dashed}\PY{l+s+s1}{\PYZsq{}}\PY{p}{)}
\PY{n}{axs}\PY{p}{[}\PY{l+m+mi}{2}\PY{p}{]}\PY{o}{.}\PY{n}{scatter}\PY{p}{(}\PY{n}{Y}\PY{p}{,}\PY{n}{Y\PYZus{}ridge\PYZus{}2}\PY{p}{)}
\PY{n}{axs}\PY{p}{[}\PY{l+m+mi}{2}\PY{p}{]}\PY{o}{.}\PY{n}{plot}\PY{p}{(}\PY{n}{lims}\PY{p}{,}\PY{n}{lims}\PY{p}{,}\PY{n}{alpha}\PY{o}{=}\PY{l+m+mf}{0.2}\PY{p}{,}\PY{n}{color}\PY{o}{=}\PY{l+s+s1}{\PYZsq{}}\PY{l+s+s1}{black}\PY{l+s+s1}{\PYZsq{}}\PY{p}{,}\PY{n}{linestyle}\PY{o}{=}\PY{l+s+s1}{\PYZsq{}}\PY{l+s+s1}{dashed}\PY{l+s+s1}{\PYZsq{}}\PY{p}{)}
\PY{n}{axs}\PY{p}{[}\PY{l+m+mi}{0}\PY{p}{]}\PY{o}{.}\PY{n}{set\PYZus{}ylabel}\PY{p}{(}\PY{l+s+s1}{\PYZsq{}}\PY{l+s+s1}{Horas de trabajo/mes (predicho)}\PY{l+s+s1}{\PYZsq{}}\PY{p}{)}
\PY{n}{axs}\PY{p}{[}\PY{l+m+mi}{1}\PY{p}{]}\PY{o}{.}\PY{n}{set\PYZus{}xlabel}\PY{p}{(}\PY{l+s+s1}{\PYZsq{}}\PY{l+s+s1}{Horas de trabajo/mes (real)}\PY{l+s+s1}{\PYZsq{}}\PY{p}{)}
\PY{n}{axs}\PY{p}{[}\PY{l+m+mi}{0}\PY{p}{]}\PY{o}{.}\PY{n}{set\PYZus{}title}\PY{p}{(}\PY{l+s+s1}{\PYZsq{}}\PY{l+s+s1}{Ridge con heurística}\PY{l+s+s1}{\PYZsq{}}\PY{p}{)}
\PY{n}{axs}\PY{p}{[}\PY{l+m+mi}{1}\PY{p}{]}\PY{o}{.}\PY{n}{set\PYZus{}title}\PY{p}{(}\PY{l+s+s1}{\PYZsq{}}\PY{l+s+s1}{Ridge con LOOCV}\PY{l+s+s1}{\PYZsq{}}\PY{p}{)}
\PY{n}{axs}\PY{p}{[}\PY{l+m+mi}{2}\PY{p}{]}\PY{o}{.}\PY{n}{set\PYZus{}title}\PY{p}{(}\PY{l+s+s1}{\PYZsq{}}\PY{l+s+s1}{Ridge con 5\PYZhy{}Fold CV}\PY{l+s+s1}{\PYZsq{}}\PY{p}{)}
\end{Verbatim}
\end{tcolorbox}

            \begin{tcolorbox}[breakable, size=fbox, boxrule=.5pt, pad at break*=1mm, opacityfill=0]
\prompt{Out}{outcolor}{13}{\boxspacing}
\begin{Verbatim}[commandchars=\\\{\}]
Text(0.5, 1.0, 'Ridge con 5-Fold CV')
\end{Verbatim}
\end{tcolorbox}
        
    \begin{center}
    \adjustimage{max size={0.9\linewidth}{0.9\paperheight}}{Tarea1_JLAC_files/Tarea1_JLAC_36_1.png}
    \end{center}
    { \hspace*{\fill} \\}
    
    De estas gráficas puede observarse que las predicciones tienden a estar
cerca de la línea identidad, lo cual sugiere que el modelo se acerca a
las predicciones esperadas. Se observa nuevamente que probablemente los
4 valores más elevados pueden tender a ser muy influyentes en los
parámetros de la regresión, pues se encuentran muy separados del resto
de los valores. \#\#\#\#\# Inciso 5. Se respondió a la vez que el inciso
3, para mostrar los valores de los parámetros. \#\#\#\#\# Inciso 6. A
continuación se muestran los factores de inflación de varianza
calculados, tanto para un modelo lineal ordinario como para el Ridge,
con las siguientes ecuaciones:
\[VIF_j=n(\mathbf{X}_c^T\mathbf{X}_c)^{-1}_{jj},j=\{1,2,...,p\}\] Para
los VIF de Ridge:
\[VIF_j=n\left[(\mathbf{X}_c^T\mathbf{X}_c+\lambda I)^{-1}\mathbf{X}_c^T\mathbf{X}_c(\mathbf{X}_c^T\mathbf{X}_c+\lambda I)^{-1}\right]_{jj}\]
Los VIF son los siguientes:

    \begin{tcolorbox}[breakable, size=fbox, boxrule=1pt, pad at break*=1mm,colback=cellbackground, colframe=cellborder]
\prompt{In}{incolor}{14}{\boxspacing}
\begin{Verbatim}[commandchars=\\\{\}]
\PY{c+c1}{\PYZsh{} VIF lineal}
\PY{n}{VIFs} \PY{o}{=} \PY{n}{n}\PY{o}{*}\PY{n}{np}\PY{o}{.}\PY{n}{diag}\PY{p}{(}\PY{n}{np}\PY{o}{.}\PY{n}{linalg}\PY{o}{.}\PY{n}{inv}\PY{p}{(}\PY{n}{Xc}\PY{o}{.}\PY{n}{T}\PY{n+nd}{@Xc}\PY{p}{)}\PY{p}{)}
\PY{c+c1}{\PYZsh{} VIF ridges:}
\PY{n}{lambdas} \PY{o}{=} \PY{p}{\PYZob{}}\PY{l+s+s1}{\PYZsq{}}\PY{l+s+s1}{Heurística}\PY{l+s+s1}{\PYZsq{}}\PY{p}{:} \PY{n}{lamb}\PY{p}{[}\PY{l+m+mi}{0}\PY{p}{]}\PY{p}{,} \PY{l+s+s1}{\PYZsq{}}\PY{l+s+s1}{LOOCV:}\PY{l+s+s1}{\PYZsq{}}\PY{p}{:} \PY{n}{ridge}\PY{o}{.}\PY{n}{alpha\PYZus{}}\PY{p}{,} \PY{l+s+s1}{\PYZsq{}}\PY{l+s+s1}{5\PYZhy{}Fold CV}\PY{l+s+s1}{\PYZsq{}}\PY{p}{:} \PY{n}{ridge\PYZus{}2}\PY{o}{.}\PY{n}{alpha\PYZus{}}\PY{p}{\PYZcb{}}
\PY{k}{with} \PY{n}{np}\PY{o}{.}\PY{n}{printoptions}\PY{p}{(}\PY{n}{precision}\PY{o}{=}\PY{l+m+mi}{2}\PY{p}{,}\PY{n}{suppress}\PY{o}{=}\PY{k+kc}{True}\PY{p}{)}\PY{p}{:}
    \PY{n+nb}{print}\PY{p}{(}\PY{l+s+sa}{f}\PY{l+s+s1}{\PYZsq{}}\PY{l+s+s1}{VIF lineal: }\PY{l+s+si}{\PYZob{}}\PY{n}{VIFs}\PY{l+s+si}{\PYZcb{}}\PY{l+s+s1}{\PYZsq{}}\PY{p}{)}
    \PY{k}{for} \PY{n}{value} \PY{o+ow}{in} \PY{n}{lambdas}\PY{p}{:}
        \PY{n}{mat} \PY{o}{=} \PY{n}{np}\PY{o}{.}\PY{n}{linalg}\PY{o}{.}\PY{n}{inv}\PY{p}{(}\PY{n}{Xc}\PY{o}{.}\PY{n}{T}\PY{n+nd}{@Xc} \PY{o}{+} \PY{n}{lambdas}\PY{p}{[}\PY{n}{value}\PY{p}{]}\PY{o}{*}\PY{n}{np}\PY{o}{.}\PY{n}{eye}\PY{p}{(}\PY{n}{p}\PY{p}{)}\PY{p}{)} \PY{o}{@} \PY{n}{Xc}\PY{o}{.}\PY{n}{T}\PY{n+nd}{@Xc} \PY{o}{@} \PY{n}{np}\PY{o}{.}\PY{n}{linalg}\PY{o}{.}\PY{n}{inv}\PY{p}{(}\PY{n}{Xc}\PY{o}{.}\PY{n}{T}\PY{n+nd}{@Xc} \PY{o}{+} \PY{n}{lambdas}\PY{p}{[}\PY{n}{value}\PY{p}{]}\PY{o}{*}\PY{n}{np}\PY{o}{.}\PY{n}{eye}\PY{p}{(}\PY{n}{p}\PY{p}{)}\PY{p}{)}
        \PY{n}{vals} \PY{o}{=} \PY{p}{(}\PY{n}{n}\PY{o}{*}\PY{n}{np}\PY{o}{.}\PY{n}{diag}\PY{p}{(}\PY{n}{mat}\PY{p}{)}\PY{p}{)}
        \PY{n+nb}{print}\PY{p}{(}\PY{l+s+sa}{f}\PY{l+s+s1}{\PYZsq{}}\PY{l+s+s1}{VIF de ridge }\PY{l+s+si}{\PYZob{}}\PY{n}{value}\PY{l+s+si}{\PYZcb{}}\PY{l+s+s1}{: }\PY{l+s+si}{\PYZob{}}\PY{n}{vals}\PY{l+s+si}{\PYZcb{}}\PY{l+s+s1}{\PYZsq{}}\PY{p}{)}
\end{Verbatim}
\end{tcolorbox}

    \begin{Verbatim}[commandchars=\\\{\}]
VIF lineal: [6083.13    7.95 5734.87   16.35    4.8 ]
VIF de ridge Heurística: [1.07 2.9  1.19 2.94 1.41]
VIF de ridge LOOCV:: [0.75 2.25 0.83 2.13 1.23]
VIF de ridge 5-Fold CV: [3.39 5.76 3.78 7.77 2.34]
    \end{Verbatim}

    \hypertarget{inciso-7.}{%
\subparagraph{Inciso 7.}\label{inciso-7.}}

La regresión Ridge ayuda para disminuir los problemas de colinealidad,
observado claramente con la reducción en los factores de inflación de la
varianza. Las predicciones se acercan a los valores reales, y los
residuales se aproximan a una distribución normal. Debido a que existían
algunos valores más influyentes en las distribuciones, las validaciones
cruzadas muestran valores muy diferentes dependiendo de los datos usados
para la validación. Si no se hace regresión Ridge, puede parecer que
algunas variables no son significativas para explicar las horas de
trabajo por mes, a pesar de ser muy buenos predictores, debido al
incremento de varianza en los estimadores provocados por la
colinealidad.


    % Add a bibliography block to the postdoc
    
    
    
\end{document}
