\subsection*{Tercer ejercicio}

Para el tercer ejercicio, se divide en incisos la solución. En su primer inciso,
considerando en primer lugar a $\eta(x)$, partiendo de la tabla \ref{tab:prob}:

\begin{align}
    \eta(x) &= \begin{cases}
        \frac{0.15}{0.3}, & x=0 \\
        \frac{0.2}{0.3}, & x=1 \\
        \frac{0.1}{0.4}, & x=2
    \end{cases} \\
    &= 
    \begin{cases}
        \frac{1}{2}, & x=0 \\
        \frac{2}{3}, & x=1 \\
        \frac{1}{4}, & x=2
    \end{cases}
\end{align}

Si un falso positivo tiene el mismo costo que un falso negativo, entonces la función de costo puede ser definida de la siguiente manera:
\begin{equation}
    L^* = \P(Y=1\cap g^*(X)=0) +  \P(Y=0\cap g^*(X)=1)
\end{equation}
Que es equivalente a cómo se ha descrito desde el comienzo, por lo que $g^*(x)$
sigue siendo igual. Por lo tanto, el clasificador óptimo de Bayes es el siguiente: 
\begin{equation}
    g^*(x) = 
    \begin{cases}
        0, & x=0 \\
        1, & x=1 \\
        0, & x=2
    \end{cases}\label{eq:Bayes}
\end{equation}
Que es la respuesta al primer inciso. Calculando la probabilidad de cometer un error: 
\begin{equation}
    \P(Y\ne g(X)) = \P(Y=1\cap g^*(X)=0) +  \P(Y=0\cap g^*(X)=1)
\end{equation}
Por ley de probabilidad total puede escribirse de la siguiente manera: 
\begin{align*}
    \P(Y\ne g(X)) &= \sum_{x=0}^2 \P(Y=1\cap g^*(x)=0) + \P(Y=0\cap g^*(x)=1) \\
     &= 0 + 0.15 + 0.1 + 0 + 0 + 0.1 \\
     &= 0.35
\end{align*} 
Y con esto concluye el segundo inciso del tercer ejercicio. Para el tercer inciso, 
la función de costo puede describirse de la siguiente manera: 
\begin{align}
    L^* &= \P(Y=1\cap g^*(X)=0) +  2\P(Y=0\cap g^*(X)=1) \\
        &= \E_X\left[P(Y=1|X=x)\ind[g^*(X=0)]+2P(Y=1|X=x)\ind[g^*(X=1)]\right] \\
        &= \E_X\left[\eta(X)\ind[g^*(X=0)]+2(1-\eta(X))\ind[g^*(X=1)]\right]
\end{align}
Entonces, $g^*(x)$ puede ser minimizado tomando para todo $x$:
\begin{align}
    g^*(x) &= 
    \begin{cases}
        1, & \eta(x) > 2-2\eta(x), \\
        0 & \text{en otro caso.} 
    \end{cases} \\
    &=
    \begin{cases}
        1, & \eta(x) > \frac{2}{3}, \\
        0 & \text{en otro caso.} 
    \end{cases}
\end{align}
Por lo tanto, el clasificador de Bayes óptimo es el siguiente: 
\begin{equation}
    g^*(x) = 
    \begin{cases}
        0, & x=0 \\
        0, & x=1 \\
        0, & x=2
    \end{cases}
\end{equation}
Y su costo, en este caso, sería de 0.45\footnote{No es muy complicado de mostrar; se sigue la misma lógica que en el inciso anterior. Se omite el procedimiento debido a que queda fuera del alcande de la pregunta y se menciona meramente como comentario.}. Si por el otro lado, en vez de tomar como límite 
que $\eta(x) > \frac{2}{3}$ para que lo clasifique como 1, fuera una desigualdad 
no estricta $(\eta(x) \ge \frac{2}{3})$, el clasificador de Bayes sería igual al mostrado en la ecuación \ref{eq:Bayes}, 
con un costo de 0.45 igualmente.


