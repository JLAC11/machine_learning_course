\subsection*{Segundo ejercicio}
Para el segundo ejercicio, asumamos que $\tilde{\eta}$ es meramente un estimado
de la probabilidad de que $Y$ sea 1 dado un valor de $X$, donde $X\in\mathcal{X}$. Dado el clasificador $g$, 
entonces debe considerarse no solamente el error discutido en el ejercicio anterior, 
sino que debido a que $\tilde{\eta}(X) \ne \eta(X)$. 
Sustituyendo el resultado anterior en la desigualdad: 

\begin{equation}
    2\E\left[\left|\eta(X)-\frac{1}{2}\right|\ind[g^*(X)\ne g(X)]\right]\le 2\E\left[\eta(X)-\tilde{\eta}(X)\right]
\end{equation}

Si $g(X) = g^*(X)$ entonces el lado izquierdo de la ecuación es 0, por lo que 
la desigualdad siempre se cumple. Entonces, solo debe revisarse en el caso que 
sean distintos. Si esta desigualdad se cumple para todo $X\in \{\mathcal{X}:g(X)\ne g^*(X)\}$, 
entonces se cumple para sus valores esperados. Para que $g(X)\ne g^*(X)$, debe 
cumplirse que $g(X) = 0,\ g^*(X)=1$ o $g(X) = 0, g^*(X)=1$. Es decir, exclusivamente 
debe ser uno u otro; en una notación más estándar: $\ind{g^*(X)\ne g(X)} = g^*(X)\oplus g(X)$, donde $\oplus$ se entiende como 
una disyunción exclusiva. Para que esto ocurra, entonces necesariamente:

\begin{equation}
    \left(\eta(x)\le \frac{1}{2} \land \tilde{\eta}(x) > \frac{1}{2}\right)\oplus\left(\eta(x) > \frac{1}{2} \land \tilde{\eta}(x)\le \frac{1}{2}\right) 
\end{equation}

Suponiendo el primer caso, entonces se tiene la desigualdad: 

\begin{align}
    \left|\eta(x)-\frac{1}{2}\right| &\le |\eta(X)-\tilde{\eta}(X)| \\
    \frac{1}{2}-\eta(x) &\le \tilde{\eta}(X) - \eta(X) \\
    \frac{1}{2} &\le \tilde{\eta}(X)
\end{align}

Que es verdadero para todo $X\in \{\mathcal{X}:g(X)\ne g^*(X)\}$ dado el primer caso. 
Resolviendo la desigualdad en el segundo caso: 
\begin{align}
    \left|\eta(x)-\frac{1}{2}\right| &\le |\eta(X)-\tilde{\eta}(X)| \\
    \eta(x)-\frac{1}{2} &\le  \eta(X)-\tilde{\eta}(X) \\
    \frac{1}{2} &\ge \tilde{\eta}(X)
\end{align}
Que también es verdadero para todo $X\in \{\mathcal{X}:g(X)\ne g^*(X)\}$ dado el segundo caso. 
Debido a que la desigualdad se cumple, entonces se llega al resultado: 
\begin{equation}
    2\E\left[\left|\eta(X)-\frac{1}{2}\right|\ind[g^*(X)\ne g(X)]\right]\le 2\E\left[\eta(X)-\tilde{\eta}(X)\right]
\end{equation}
Y con esto concluye la demostración. $\blacksquare$